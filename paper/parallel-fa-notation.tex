\section{Notation}
\label{sec:notation}

Because the $g$ output function is optional, we imitate the Kronecker delta
function to simplify the resource counts with a single multiplicative
constant:

\begin{equation*}
\delta_g = \left\{
  \begin{array}{rl}
    0 & \text{if } g \text{ is unused}\\
    1 & \text{if } g \text{ is used}\\
  \end{array} \right.
\end{equation*}

We denote by $w(f)$ for a particular function table $f$ to be the
number of ones that appear in the outputs. If the optional output function $g$
is not needed by an FA, then we have $w(g) = k_o = 0$.
The number of ones can be at most the number of lines in the function
table, so $w(f) \le 2^{k_i+k_s}$.

Similarly, we denote by $z(f)$ for a particular function table $f$ to be
the number of ones which appear in the outputs for a zero input. Again,
an unused function $g$ would result in $z(g) = 0$.
Naturally, we can't have more output ones for zero inputs than
output ones for any inputs: $z(f) \le w(f)$.

Since $n$ may not be a power of two, a common parameter will be the
logarithm of $n$ rounded up to the nearest power of two, which we'll
defined as $l = \lceil \log_2 n \rceil$.

In the gate counts below, we will include the counts of single-qubit NOT, or X,
gates, which is used to negate the control of controlled-NOTs or Toffoli's.

% \begin{figure}
% \begin{displaymath}
% \Qcircuit @C=2em @R=1.5em {
% & \ctrlo{1} & \qw & =  & & \qw & \targ & \ctrl{1} & \qw &\\
% & \targ     & \qw &    & & \qw & \qw   & \targ    & \qw &
% }
% \end{displaymath}
% \end{figure}

\begin{figure}
\begin{displaymath}
\Qcircuit @C=2em @R=1.5em {
& \ctrlo{1} & \qw & =  & & \qw & \targ & \ctrl{1} & \qw &\\
& \ctrl{1}     & \qw &    & & \qw & \qw   & \ctrl{1}    & \qw &\\
& \targ     & \qw &    & & \qw & \qw   & \targ    & \qw &
}
\end{displaymath}
\caption{Negative control of multiply-controlled NOT gates}
\end{figure}

To compare the relative complexity of implementing the Toffoli gate versus
CNOT and single-qubit gates, we can consider the following implementation
of Toffoli given in Figure 4.9, page 182 of \cite{nc00}. One Toffoli equals
6 CNOT gates and 10 single-qubit gates (Hadamard and controlled-phase).

We denote $d_T$ as the depth of a Toffoli gate, $d_C$ as the depth of a
CNOT, and $d$ as the depth of a single-qubit gate. That last one we admit
can vary widely based on the particular single-qubit gate, but we have to
draw the line at some point. So based on our implementation of the
Toffoli above, we have:

\begin{displaymath}
d_T = 6d_C + 10d
\end{displaymath}

We also note that a multiply ($n$) controlled-NOT ($\Lambda^n(X)$) can be
implemented with $n-1$ Toffoli's and $n-2$ ancillae, where $n=2$ is a
normal Toffoli.

In each of the four steps of parallel iteration described in Section
\ref{sec:pi}, we provide a table of resource usage: the number of gates
of each type (single-qubit NOT, CNOT, and Toffoli), the number of ancillae,
and the depth of the circuit. The number of gates is specified per input and
often redefined as $\ell$ with some subscript, which we then reuse in
the circuit depth for notational convenience. This is because
these operations happen in parallel for each of the inputs, leading to an
implicit factor $n$. However, this factor disappears or becomes logarithmic
in the circuit depth; otherwise, we would not have accomplished our goal
of decreasing depth with parallel iteration!
There is also always a factor of 2 in the depth
and gate count to allow for uncomputing the ancillae back to $\ket{0}$.
