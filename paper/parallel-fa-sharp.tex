\section{Parallelized Phase Sharpening}
\label{sec:sharp}

Another surprising application of finite automata and its parallel
iteration is in sharpening an estimate of phases (eigenvalues) of
a unitary operator.
Phase estimation is an important part of
many quantum algorithms, including Shor's algorithms for factoring and
finding discrete logarithms and solving systems of linear equations.

Phase sharpening is one of the classical postprocessing steps in Kitaev's
phase estimation procedure described on p. [to be added] of \cite{ksv02}.
The remarkable result of phase sharpening is that given many modest
estimates (say within $1/8$) of a phase $\phi$ and its multiples $2^{j}\phi$
by $2n$ successive powers of two, we can ``sharpen'' this result into
an estimate of $\phi$ of exponential precision using only a finite automata.

That is, at every step of this procedure, we only need to compare a
constant amount of bits among these modest estimates to get our
arbitrarily precise result.

\subsection{The Finite Automata Function Tables}

\renewcommand{\arraystretch}{1.2}
\begin{center}
\begin{table}
\begin{tabular}{|c|c|c||c|c||c|}
\hline
$\beta_j$ & $\alpha_{j+1}$ & $\alpha_{j+2}$ & $\overline{\beta_j}$ & $\overline{\alpha_j\alpha_{j+1}\alpha_{j+2}}$ & $\alpha_j$\\
\hline
000 & 0 & 0 & $\frac{0}{8}$ & $\frac{0}{8}$ & 0\\
    & 0 & 1 &               & $\frac{1}{8}$ & 0\\
    & 1 & 0 &               &               & X\\
    & 1 & 1 &               & $\frac{3}{8}$ & 0\\
\hline
001 & 0 & 0 & $\frac{1}{8}$ & $\frac{0}{8}$ & 0\\
    & 0 & 1 &               & $\frac{1}{8}$ & 0\\
    & 1 & 0 &               & $\frac{2}{8}$ & 0\\
    & 1 & 1 &               &               & X\\
\hline
010 & 0 & 0 & $\frac{2}{8}$ &               & X\\
    & 0 & 1 &               & $\frac{1}{8}$ & 0\\
    & 1 & 0 &               & $\frac{2}{8}$ & 0\\
    & 1 & 1 &               & $\frac{3}{8}$ & 0\\
\hline
011 & 0 & 0 & $\frac{3}{8}$ & $\frac{4}{8}$ & 1\\
    & 0 & 1 &               &               & X\\
    & 1 & 0 &               & $\frac{2}{8}$ & 0\\
    & 1 & 1 &               & $\frac{3}{8}$ & 0\\
\hline
100 & 0 & 0 & $\frac{4}{8}$ & $\frac{4}{8}$ & 1\\
    & 0 & 1 &               & $\frac{5}{8}$ & 1\\
    & 1 & 0 &               &               & X\\
    & 1 & 1 &               & $\frac{3}{8}$ & 0\\
\hline
101 & 0 & 0 & $\frac{5}{8}$ & $\frac{4}{8}$ & 1\\
    & 0 & 1 &               & $\frac{5}{8}$ & 1\\
    & 1 & 0 &               & $\frac{6}{8}$ & 1\\
    & 1 & 1 &               &               & X\\
\hline
110 & 0 & 0 & $\frac{6}{8}$ &               & X\\
    & 0 & 1 &               & $\frac{5}{8}$ & 1\\
    & 1 & 0 &               & $\frac{6}{8}$ & 1\\
    & 1 & 1 &               & $\frac{7}{8}$ & 1\\
\hline
111 & 0 & 0 & $\frac{7}{8}$ & $\frac{0}{8}$ & 0\\
    & 0 & 1 &               &               & X\\
    & 1 & 0 &               & $\frac{6}{8}$ & 1\\
    & 1 & 1 &               & $\frac{7}{8}$ & 1\\
\hline
\end{tabular}
\caption{Transition function table for phase sharpening}
\end{table}
\end{center}