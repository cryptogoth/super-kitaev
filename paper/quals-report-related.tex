\section{Related Work}

In 1995, Seth Lloyd found that almost any two distinct single-qubit rotations are
universal for approximating an arbitrary single-qubit rotation, but that this
approximation could be exponentially long in both time and length $T,L = (O(1/\epsilon))$ \cite{Lloyd1995}.

The theorem which is now called Solovay-Kitaev was discovered by Solovay in
1995 in an unpublished manuscript and independently later discovered by
Kitaev in 1997 \cite{nc00} which showed that $T,L = O(\log^c{1/\epsilon})$ for
$c$ between 3 and 4.

In 2001, Aram completed his undergrad thesis arguing that it would be difficult
to beat $c < 2$ for the above bounds using a successive approximation method.\cite{harrow01}

In 2002, Kitaev, Shen, and Vyalyi published their book which contains an
application of parallelized phase estimation towards simulating a quantum
circuit (what we are calling Super-Kitaev) \cite{ksv02}.
That is, an alternative quantum compiler to Solovay-Kitaev which has
asymptotically better circuit size and depth, but using ancillary qubits
and having as yet unknown multiplicative constants.

In 2003, Aram, Ben Recht, and Ike demonstrated that a certain universal
set could be used to saturate the lower bound $L=O(\log{1/\epsilon})$
but it remains an open problem whether any efficient algorithm exists
which can do this in tractable $T$ \cite{hrc02}.

In 2005, Chris Dawson and Mike Nielsen published their pedagogical review
paper which I used as the basis for my previous work on implementing a
Solovay-Kitaev compiler \cite{Dawson2005}.
