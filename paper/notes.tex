\documentclass{article}

\usepackage{times}
\usepackage{fullpage}
\usepackage{eprint}
\usepackage{rotating}
\usepackage{eepic}
\usepackage{amsfonts}
\usepackage{algorithmic}
\usepackage{amsthm}

%\tocdetails{%
%   volume=0, number=5, year=2005, firstpage=78,
%   received={October 15, 2004}, 
%   published={December 31, 2004}
%}

\theoremstyle{plain}
\newtheorem{theorem}{Theorem}
%\newtheorem{idea}{Idea}[section]

%\theoremstyle{definition}
%\newtheorem{thought}[theorem]{Thought}
%\newtheorem{thought}{Thought}[section]

\title{Quantum Compiling with the Super-Kitaev Algorithm}
\date{November 2nd 2010}
\author{Paul Pham}

\input{Qcircuit}

\begin{document}

%\newcommand{\ket}[1]{|#1 \rangle}
%\newcommand{\bra}[1]{\langle #1 |}
\newcommand{\braket}[2]{\langle #1|#2 \rangle}
\newcommand{\normtwo}{\frac{1}{\sqrt{2}}}
\newcommand{\norm}[1]{\parallel #1 \parallel}

\maketitle

\section{Introduction}

When describing a quantum algorithm, we use a high-level formalism of
gates operating on qubits in the quantum circuit model.
We can treat gates operating on an n-qubit quantum computer as unitary
matrices of dimension $2^n \times 2^n$ with unit determinant.
However, in experimental settings,
we can only perform some gates efficiently, and these are usually local,
2-qubit operations. Moreover, most of our results for fault-tolerant
quantum computing in the presence of noise stipulates that we have a finite
number of universal gates we can perform with some limited precision.

How then, do we translate our quantum algorithms into something that can run
on hardware? In analogy to classical computers, we describe this translation
as {\em quantum compiling}. One of the central results of quantum computing,
and the reason why it is interesting, is that we can actually perform this
quantum compiling efficiently. This first result is called the
Solovay-Kitaev theorem \cite{Dawson2005}, which provides an efficient way to
approximate any quantum gate. I presented this algorithm in an earlier talk, but
I'll briefly review the results here. However, the constants for this method
scale exponentially with the number of qubits and involve an intractably
large preprocessing step.

The second result is more recent but much less well-known, and is called
the Super-Kitaev algorithm (named by Aram Harrow). This new method improves upon
the bounds for compiled circuit size over the normal Solovay-Kitaev algorithm,
but comparison is more complicated. Super-Kitaev introduces some new tradeoffs to the
comparison, such as ancillary qubits and classical postprocessing. In doing so,
Kitaev also gives us some interesting modules which are worth studying for
their own sake. In these notes, we will outline the Super-Kitaev algorithm at
a high-level to develop some intuition about why it works, describe how we 
might implement it in practice.

\section{Definitions}

\subsection{Approximation vs. Simulation}

I'll use the terms "approximation" and "simulation" interchangeably, as in
a quantum compiler will take an input circuit and produce as output a new
circuit which approximates or simulates the original circuit to some degree
of precision.

\subsection{Standard Set}

We use the following standard set $\mathcal{Q}$ from \cite{ksv2002}

\begin{equation}
\mathcal{Q} = \{ H, K, K^{-1}, \Lambda(\sigma^x), \Lambda^2(\sigma^x) \}
\end{equation}

where these gates are defined as follows:

\begin{equation*}
H = \normtwo \left(
  \begin{array}{cc}
    1 & 1 \\
    1 & -1 \\
  \end{array} \right)
\end{equation*}

\begin{equation*}
K = \left(
  \begin{array}{cc}
    1 & 0 \\
    0 & i \\
  \end{array} \right)
\end{equation*}

Even though realistic implementations of a quantum computer may have a different
set of hardware instructions, Kitaev's analysis assumes that the standard set
can always be efficiently compiled down further into this hardware set.

Note that this may not be valid. For example, in order for ion traps to perform
even a Hadamard gate to any precision requires a combination of $\sigma^x$ and
$\sigma^z$ gates. However, we won't let this bother us for now.

\subsection{Parameters}

For these notes, we restrict ourselves to $d=2^n$ for an $n$-qubit system (qudits)
and consider the problem of compiling an entire circuit $C$ of $L$ gates with depth
$d$ to a new, compiled circuit $C'$ of size $L'$ and depth $d'$ which approximates
$C$ within some error. There is some overhead in the compiled circuit, so in
general $C'$ is larger (that is, $L' > L$ and $d' > d$). It's also known that
in order to approximate a circuit with $L$ gates to a total precision of $\epsilon$
requires each gate to be approximated to a precision of $L/\epsilon$, which requires
the following number of gates. We'll find this $l$ parameter useful in describing the compiler overheads of
both the Solovay-Kitaev and Super-Kitaev methods.

Why do we care about depth? This gives us a heuristic for how ``parallelized'' our
circuit is. If we flatten our circuit into layers $\{l_i\}$, where each layer $l_i$
is dependent only on inputs from the previous layer $l_{i-1}$ and produces
only outputs to the next layer $l_{i+1}$, then all the gates within the layer
$l_i$ can be performed in parallel. All other things being equal, a circuit with low depth will complete
faster than one with high depth, although in practice we can only execute
fixed-width circuits.

Also, when running the compiler, there are several time resources to consider.
The first is classical preprocessing time, denoted $T_{pre}$, which is a way of optimizing
the generated quantum circuit by spending more classical resources upfront.
The second is classical postprocessing time, denoted $T_{post}$, which likewise
is meant to reduce the size of the quantum circuit but depends on quantum
measurements and may be fed back into future quantum operations later.

A summary of these parameters are provided below:

\begin{tabular}{|c|c|}
\hline\\
$d$ & dimension of quantum gates\\
$n$ & number of qubits, $d=2^n$\\
$\epsilon$ & the desired accuracy of the compiled circuit\\
$L$ & size of input circuit to be compiled, in gates from $\mathcal{Q}$\\
$d$ & depth of input circuit to be compiled\\
$L'$ & size of compiled output circuit, in gates from $\mathcal{Q}$\\
$d'$ & depth of compiled circuit to be compiled\\
$l$ & $=O(\log{L/\epsilon})$, shorthand parameter for asymptotic analysis\\
$T$ & classical running time of the compiling algorithm\\
$T_{pre}$ & classical preprocessing time, before the compiler is run\\
$T_{post}$ & classical postprocessing time, after the compiler is run or during\\
\hline\\
\end{tabular}

\subsection{Universal Instruction Sets and Compiled Sequences}
More formally, suppose we have a universal instruction set $\mathcal{G}$ which
contains some finite number of gates $g \in SU(d)$ and also their inverses $g^\dagger$.
$\mathcal{G}$ is universal for $SU(d)$ in that it generates a dense subgroup.
In other words, given an arbitrary quantum gate $U \in SU(d)$ and a desired
accuracy $\epsilon$ we can find a product $S=g_1 \cdots g_m$ from $\mathcal{G}$
which approximates $U$ to within $\epsilon$ using some distance function.

We can either use the matrix operator norm to define the distance function
such as:

\begin{equation}
d(U,S) \equiv \norm{U - S} \equiv \sup_{\norm{\ket{\psi}}=1} \norm{(U-S)\ket{\psi}} < \epsilon
\end{equation}

There is also a trace measure introduced by Austin Fowler which disregards
the global phase factor, so that we don't waste time trying to approximate
the unmeasurable phase of our target gate.

\begin{equation}
d(U,S) = \sqrt{\frac{d - \norm{\mathrm{tr}(U^\dagger S)}}{d}}
\end{equation}

\section{History}

The theorem which is now called Solovay-Kitaev was discovered by Solovay in
1995 in an unpublished manuscript and independently later discovered by
Kitaev in 1997 \cite{nc00}.

Also in 1995, Seth Lloyd found that any two distinct single-qubit rotations are
universal for approximating an arbitrary single-qubit rotation, but that this
approximation could be exponentially long in both time and length.

In 200

In 2002, Kitaev, Shen, and Vyalyi published their book which contains an
application of parallelized phase estimation towards simulating a quantum
circuit (what we are calling Super-Kitaev).
That is, an alternative quantum compiler to Solovay-Kitaev which has
asymptotically better circuit size and depth, but using ancillary qubits
and having as yet unknown multiplicative constants.

In 2003, Aram, Ben Recht, and Ike showed a lower bound on any Solovay-Kitaev
style approximation of $T,L = O(\log{1/\epsilon})$.

In 2005, Chris Dawson and Mike Nielsen published their pedagogical review
paper which I used as the basis for my previous work on implementing a
Solovay-Kitaev compiler.

\section{Review of Solovay-Kitaev}

Here I'll remind you of the basic pseudo-code for normal Solovay-Kitaev.
Our development is taken from the excellent review paper \cite{Dawson2005}.

\begin{algorithmic}[1]
\STATE \textsc{function} $\tilde{U}_n \leftarrow$ SOLOVAY-KITAEV$(U,n)$
\IF{$n == 0$}
\STATE $\tilde{U}_n \leftarrow $ BASIC-APPROX$(U)$
\ELSE
\STATE $\tilde{U}_{n-1} \leftarrow$ SOLOVAY-KITAEV$(U, n-1)$
\STATE $A,B \leftarrow $ FACTOR$(U\tilde{U}^\dagger_{n-1})$
\STATE $\tilde{A}_{n-1} \leftarrow $ SOLOVAY-KITAEV$(A, n-1)$
\STATE $\tilde{B}_{n-1} \leftarrow $ SOLOVAY-KITAEV$(B, n-1)$
\STATE $\tilde{U}_n \leftarrow \tilde{A}_{n-1}\tilde{B}_{n-1}\tilde{A}^\dagger_{n-1}\tilde{B}^\dagger_{n-1}\tilde{U}_{n-1}$
\ENDIF
\RETURN $\tilde{U}_n$
\end{algorithmic}

This algorithm works by way of recursive, successive approximation.

The BASIC-APPROX function above does a lookup (via some kd-tree search
maneuvers through higher-dimensional vector spaces) using the results of
precompiled sequences from the instruction set $\mathcal{G}$. This can be
done offline and reused across multiple runs of the compiler, assuming
$\mathcal{G}$ for your quantum computer doesn't change.

The FACTOR function performs a balanced group commutator decomposition,
$U = ABA^\dagger B^\dagger$, and then recursively approximates the $A$ and $B$
operators using Solovay-Kitaev. Intuitively, when they are multiplied
together again, along with their inverses, their errors (which go like
$\epsilon$) are symmetric and cancel out in such a way that the resulting
product $U$ has errors which go like $\epsilon^2$. In this manner, we can
eventually sharpen our desired error down to any value.

And that's all I'm going to say about that.

\section{The Super-Kitaev Algorithm}

Our development of the Super-Kitaev algorithm follows closely the one in the
book by Kitaev, Shen, and Vyalvi \cite{ksv2002}, although of course, it is
simply called Theorem 13.5. Likewise, we will cite the theorem / lemma numbers
from the book, which contains a self-contained description of all the parts
needed for Super-Kitaev.

The basic steps of this method are the following:

\begin{enumerate}
\item Precompiling to get $L'$ gates in $\mathcal{Q} \cup \Lambda(e^{i\phi})$.
Here we see that there is an
efficient decomposition of several important gates into the standard set and
controlled-phase-shifts.
\item Implementing a controlled-phase-shift. This ends up being the hardest
part of simulating a quantum circuit, and requires each of the remaining steps.
\item Use phase estimation of an addition operator to magic states in order to
enact the desired phase shift. This requires both efficient creation of
{\em magic states} and an efficient {\em quantum adder} circuit.
\item Create one magic state, and then make $L'$ copies of it. Use one copy
to simulate each $\Lambda(e^{i\phi})$ gate.
\end{enumerate}

\subsection{Precompiling to the Standard Set}
\label{subsec:precompile}

In order to perform the precompilation step, we need to know that we can
change any unitary in $SU(d)$ by tensor products of unitaries in $SU(2)$
and $SU(4)$, that is, single- and two-qubit gates.

To prove the universality of the standard set $\mathcal{Q}$ is beyond the
scope of these notes. For our purposes, we assume that we can precompile
any gate down to single-qubit rotations $U \in SU(2)$ and
controlled-single-qubit rotations $\Lambda(U), U\in SU(2)$.
(For example, it's know how to do that for the Toffoli gate, which is what we
need to implement our quantum adder circuit).
From there, how do we get down to the standard set?

%\begin{theorem}[Lemma 8.2]
%\label{lemma82}
%Any arbitrary unitary operator $U$ of dimension $d \times d$ can be represented
%as a product of $d(d-2)/2$ matrices of the form:
%
%\begin{equation*}
%\left( \begin{array}{cccccccc}
%     1 & 0      & \cdots & \cdots & \cdots & \cdots & \cdots & \cdots\\
%     0 & 1      & \cdots & \cdots & \cdots & \cdots & \cdots & \cdots\\
%\vdots & \cdots & \ddots & \cdots & \cdots & \cdots & \cdots & \cdots\\
%\vdots & \cdots & \cdots &      a &      b & \cdots & \cdots & \cdots\\
%\vdots & \cdots & \cdots &      c &      d & \cdots & \cdots & \cdots\\
%\vdots & \cdots & \cdots & \cdots & \cdots & \vdots & \cdots & \cdots\\
%\vdots & \cdots & \cdots & \cdots & \cdots & \cdots & 1 & 0\\
%\vdots & \cdots & \cdots & \cdots & \cdots & \cdots & 0 & 1\\
%\end{array}
%\right)
%\end{equation*}
%
%where $\left( \begin{array}{cc}
%a & b\\
%c & d\\
%\end{array} \right) \in SU(2)$
%\end{theorem}
%
%\begin{proof}
%The running time of this algorithm is $O(M^3) \cdot \mathrm{poly}(\log(1/\delta)$.
%\end{proof}

\begin{theorem}[Problem 8.1]
Any single-qubit gate $U$ can be simulated using $\mathcal{Q} \cup \{ \Lambda(e^{i\phi}) \}$.
\end{theorem}

\begin{proof}
Using the Euler angle decomposition into $X$ and $Z$ rotations, we can express any
unitary $U$ as follows, with real angles $\phi$, $\gamma$, $\beta$, $\alpha$:

\begin{equation}
U = e^{i\phi}e^{i(\gamma/2)\sigma^z}e^{i(\beta/2)\sigma^x}e^{i(\alpha/2)\sigma^z}
\end{equation}
This involves solving four equations in four variables and can be done in constant time.

Now how do we implement each of the elements of the Euler angle decomposition from
the $\mathcal{Q}$? Using these identities:

\begin{equation}
e^{i\phi} = \Lambda(e^{i\phi})\sigma^x\Lambda(e^{i\phi})\sigma^x
\end{equation}
\begin{equation}
\sigma^x = H\Lambda(e^{i\pi})H
\end{equation}
\begin{equation}
e^{i\phi\sigma^z} = \Lambda(e^{-i\phi)\sigma^x\Lambda(e^{i\phi)\sigma^x
\end{equation}
\begin{equation}
e^{i\phi\sigma^x} = H e^{i\phi\sigma^z} H
\end{equation}
\end{proof}

\begin{theorem}[Problem 8.2]
Any controlled operator $\Lambda(U)$ where $U \in SU(2)$ can be simulated using
$\mathcal{Q} \cup \{ \Lambda(e^{i\phi})$.
\end{theorem}

\begin{proof}
We can represent the controlled operator as $\Lambda(U) = \Lambda(e^{i\phi})\Lambda(V)$.
$\Lambda(e^{i\phi})$ is already in our set, so it remains to implement
$\Lambda(V)$, where $V \in SU(2)$. Geometrically, we can decompose any
three-dimensional rotation into two $180^{\circ}$ rotations about appropriate
axes. Using the homomorphism between $SO(3)$ and $SU(2)$ and the fact that
$\sigma^x$ corresponds to a $180^{\circ}$ rotation about the $x$-axis.

\begin{equation}
V = A\sigma^x A^{-1} B\sigma^x B^{-1}
\end{equation}
\end{proof}

\subsection{Controlled-Phase Shift}
\label{subsec:phase-shift}

Now we are down to the standard set $\mathcal{Q}$, plus that pesky
controlled-phase-shift gate $\Lambda{e^{i\phi}}$. This is in general a hard
gate to implement, and one of the reasons why we need a quantum compiler in
the first place!

We could just use normal Solovay-Kitaev and find some sequence of $X$ and
$Z$ gates which implement $\Lambda{e^{i\phi}}$, but that wouldn't be any
fun and we would have no reason to bring in all this extra machinery to
do Super-Kitaev.

To get started, we will need to introduce a magic state $\ket{\psi}$ with the
parameters $n$ and $k$ such that

\begin{equation}
\ket{\psi_{n,k}} = \frac{1}{\sqrt{2^n}-1} \sum_{j=0}^{2^n-1} e^{2\pi i j k / 2^n} \ket{j}
\end{equation}

where $0 \le k \lt 2^n$. If you think this looks suspiciously like a
Quantum Fourier Transform state of the computational basis, you are correct.
However that doesn't help us here, because QFT is way too complicated to use as
our basic compiler. What we need instead is to make use of the index-dependent
phases of each state, and addition to move to the right phase that we want,
and easy creation of a superposition of these magic states.




\subsection{Phase Estimation}
\label{subsec:phase-estimate}

\section{Conclusion and Future Extensions}
\label{sec:conclude}

In this work, we have adapted one of the central results of quantum
computing, Grover's search algorithm, to two prominent problems of
classical relational databases. In doing so, we have designed the novel
concept of a quantum RAM and simulated a generalized quantum search
algorithm successfully.
However, this work is primarily limited by the current experimental
nature of quantum computers, and it is uncertain whether an efficient
QRAM can ever be built. In the theoretical realm, however, quantum search
can immediately be extended to aggregate counting functions, duplicate
elimination, and query size estimation.
Much more simulation work remains to be done using larger,
more realistic datasets, especially for the theoretical
problems studied in this project.
In conclusion, the author wishes to thank Dave Bacon for helpful advice
during the
course of this project.

\bibliography{report}
\bibliographystyle{ieee}

\end{document}
