\documentclass{article}

\input{QCircuit}

\title{Measurement Error in KSV Phase Estimation}

\author{Paul Pham}

\begin{document}

\maketitle

In this note, we provide a detailed calculation for the error in
phase estimation due to Kitaev, Shen, and Vyalyi (KSV) where projective
measurement is used instead of a coherent measurement. We show an
increase from $2\sqrt{\epsilon}$ to $\sqrt{2}\sqrt[4]{\epsilon}$. The use
of projective measurement is to offload many trigonometric operations
onto a classical controller instead of doing them reversibly on a quantum
computer. The cost of projective measurement is leaving garbage qubits in
the ancillae of the target register, but this is a constant amount relative to
the size of circuits we may wish to compile using the KSV procedure.
Our goal is to show that this increase in error by a square root
factor is negligible and may be preferrable in realistic implementations.

To begin with, we review measurement operators as a generalization of
controlled quantum operators, and then we further extend them to the case
of approximately measuring functions on orthogonal basis decompositions
where there are garbage bits left over in an ancillary register.
Next, we calculate the error of this approximate measurement with garbage.
Finally, we show how this general measurement procedure corresponds to
estimating the phase of the modular multiplication operator used in
Shor's factoring algorithm.

\section{Measurement Operators}

First we introduce some preliminary definitions related to measurement.

A very common quantum operation entangles the results of one register
(called the \emph{target}) based on the value of another register (called the
\emph{control}). The most basic case of this is CNOT, or $\Lambda(X)$,
operator, which operates
on a control register of one qubit and a target register of one qubit.

\begin{equation}
\Lambda(X)\ket{y,z} \rightarrow \ket{y, z \oplus y}
\end{equation}

We have two ways of describing how CNOT is measuring in this case.
We can say CNOT is \emph{measuring with respect to the decomposition} of
the control registers, namely the computational basis, in that the
operation on the target register (flipping the bit) depend on decomposing the
control register in the basis $\{0,1\}$. We can also say that CNOT is
\emph{measuring a function} $f:\{0,1\} \rightarrow \{0,1\}$ which in this
case is simply the copy operator, $f(x) = x$. We can rewrite the equation
above as:

\begin{equation}
\Lambda(X)\ket{y,z} \rightarrow \ket{y, z \oplus f(y)}
\end{equation}

However, we know that the distinction between control and target registers
often depends on a particular basis. For example, we can flip the direction
of control and target in the CNOT case by conjugating both qubits with
Hadamard operators. The general characteristic of measurement operators
is that they are entangling, and that they can encode information about
one register in another in a very general way.
From the point of view of measurement, we call the control register the
\emph{object}, as in the state we are trying to measure, and the target register
is the \emph{instrument}, as in the state that we transform according to
projecting the measurement object in some orthogonal decomposition.

Let's begin with a simple but more general case of a measuring operator
$W$ which operates on a space decomposed into the subsystems $\mathcal{N}$
(the measurement object) and $\mathcal{K}$ (the measurement instrument)
according to the orthogonal decomposition
$\mathcal{N} = \Bigoplus_j \mathcal{L}_j$, so-called because
each of the $\mathcal{L}_j$ are pairwise orthogonal subspaces.
$W$ will perform a different unitary
operator $U_j$ on the subsystem $\mathcal{K}$
depending on the projection of $\mathcal{N}$ into each $\mathcal{L}_j$.

\begin{equation}
W = \sum_{j \in \Omega} \Pi_{\mathcal{L}_j} \otimes R_j
\end{equation}

An interesting fact about this definition of measurement operators is
that approximativeness is preserved. If each unitary $U_j$ is replaced with
another unitary $\tilde{U}_j$ that approximates it with precision $\delta$,
then the new measuring operator $\tilde{W}$ also approximates the original
$W$ with the precision $\delta$.

This precision is defined in terms of the inner product between any state operated
on by $W$ and by $\tilde{W}$:

\begin{equation}

\end{equation}

\end{document}