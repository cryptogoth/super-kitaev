\section{Parallel Iterations}
\label{sec:pi}

The process of parallelizing iterations has four main steps, each of which
has its own section below with general resource requirements.

The first step is copying function tables due to fanout restrictions.
For classical circuits this is a polite gesture towards the fact that
transistors can only drive a finite (but fairly large) load. For quantum
circuits, this is an absolute necessity. So if we are going to be applying
a bunch of FA transition functions in parallel, we'd better have a separate
copy of the function for each execution.

The second step is narrowing the function tables by input. Since we know all
the input symbols in advance, we can vastly simplify the table for each
iteration $i$ by only selecting the lines which correspond to the input
symbol $a_i$. In a sense, we are ``optimizing'' the table, or hard-coding
for a particular input symbol by ``baking'' it into the execution of this new,
narrower function.

The third step is composing adjacent pairs of these narrowed function tables
in a binary tree hierarchy, so that we can skip ahead to future iterations.

Finally, the last step is to apply these composed function tables, in the
same binary tree hierarchy, to compute all the states $s_i$ in parallel
using only logarithmic depth.

\subsection{Copying Input Bits}
\label{subsec:copy}

This is the simplest step and results in $n$ copies of the input bits. Since
these are the control bits for many subsequent operations, if we wish to
source them in parallel, we must have multiple copies. This simply
involves a CNOT to copy each input into an empty ancilla.

We copy in a parallel fashion to achieve logarithmic depth, and each new
copy can itself produce another copy simultaneously. Therefore the circuit depth
is just the height of
a binary tree with $n$ nodes, times two for uncopying later.

\begin{table}
\begin{center}
\begin{tabular}{|r|cc|}
\hline
CNOT Gates Per Input &  $\ell_C = $ & $k_i \cdot 2$ \\
Ancillae & & $n\cdot k_i$\\
Circuit Depth & & $(\lceil \log_2 n \rceil - 1 )\cdot \ell_C \cdot d_C$\\
\hline
\end{tabular}
\caption{Resource counts for Step 1: copying input bits in parallel.}
\end{center}
\end{table}


\subsection{Narrowing Function Tables by Inputs}
\label{subsec:narrow}

Now that we have copied enough inputs to source, we can use them
to narrow our function tables (both $f$ and $g$) by successively
halving the function tables based on the value of each
input bit. This is where we rely on our classical control to encode
the function tables in the Toffoli's applied in the narrowings. The function
table itself appears nowhere else in our circuits.

Therefore, each of the $k_i$ input bits takes two Toffoli's,
one of which has a NOT gate to negate its control bit. This
successive halving takes the following number of ancillae, since we start
with a function table of $2^{k_i + k_s}$ and want to end with a function
table of $2^{k_s}$:

\begin{displaymath}
2^{k_i + k_s - 1} + 2^{k_i + k_s - 2} + \ldots + 2^{k_s + 1} + 2^{k_s}
\end{displaymath}

which we can express as the difference of sums:

\begin{displaymath}
\sum_{j=0}^{k_i + k_s - 1} 2^j - \sum_{j=0}^{k_s} 2^j = 
\left( 2^{k_i+k_s} - 1 \right) - \left( 2^{k_s+1} - 1 \right)
\end{displaymath}

Likewise if we use the output function $g$, we can treat each of its
$k_o$, output bits as a separate one-bit output function, and then we
require the same number of gates and ancillae as above for each one.

There is one such narrowed table copy of $f$ (and $g$) for each of $n$ inputs.
For each copy of the table, for each of $k_i$ successive narrowings and
input bits, there are several gates that need to be applied serially because
they use the same control bits---this makes up the circuit depth.
These serial gates are one Toffoli for each of $w(\cdot)$ one outputs,
$z(\cdot)$ of which require single-qubit NOT gates to negate their control.
There is also a factor
of 2 for uncomputing all of these ancillae back to $\ket{0}$.

\begin{table}
\begin{center}
\begin{tabular}{|r|cc|}
\hline
NOT Gates per Input & $\ell_N = $ & $[z(f)+z(g)] \cdot (2^{k_i+k_s} - 2^{k_s+1}) \cdot 2$ \\
Toffoli Gates per Input & $\ell_T = $ & $[w(f)+w(g)] \cdot (2^{k_i+k_s} - 2^{k_s+1}) \cdot 2$\\
Ancillae & & $n\cdot (2^{k_i+k_s} - 2^{k_s+1}) \cdot (1 + \delta_g2\cdot k_o) \cdot 2$\\
Circuit Depth & & $\ell_T d_T + \ell_N d$\\
\hline
\end{tabular}
\caption{Resource counts for Step 2: narrowing function tables by inputs in parallel}
\end{center}
\end{table}

\subsection{Composing Narrowed Function Tables}
\label{subsec:compose}

Now that we have narrowed the function tables by input, they only depend
on the current state. By composing adjacent function tables in a binary
tree hierarchy, we can compute all the intermediate states of an FA
execution while still preserving logarithmic depth.

We will use the notation $F_{k,i}$ to denote the $i$th composed function table
at the $k$th level of recursion, according to the diagram below for $n=4$.
$F_{0,i}$, at the base level, corresponds to the original function $f$
narrowed by input $a_i$.

\begin{displaymath}
\xymatrix @C=1em {
  \ar[rr]^{s_0} & &
   *+[F]{F_{0,0}} \ar[rr]^{s_1} & &
   *+[F.]{F_{0,1}} \ar[rr]^{s_2} & &
   *+[F]{F_{0,2}} \ar[rr]^{s_3} & &
   *+[F.]{F_{0,3}} \ar[rr]^{s_4} & & \\
  & & & *+[F]{F_{1,0}} \ar@{<-}[ul] \ar@{<-}[ur]
  & & & & *+[F.]{F_{1,1}} \ar@{<-}[ul] \ar@{<-}[ur] \\
  & & & & & *+[F]{F_{2,0}} \ar@{<-}[ull] \ar@{<-}[urr]
}
\end{displaymath}

How many composition operations do we need to do for our $n$ initial
narrowed $F_{0,i}$ tables? The resulting full binary tree (except for
the first level) has $\lceil \log_2 n \rceil = l$ levels and therefore
$2^{l}-1$ nodes. However, as will become apparent in the next section, we don't
need the compositions with a dotted border above.

Since at each level of compositions, we can remove half of the nodes
(except for the root node), we have the following formula of the
total number of compositions we need for $n$ iterations and $l$ levels,
where the root level corresponds to $j=0$.

\begin{equation}
\sum_{j=0}^{l-1} 2^j - \sum_{j=1}^{l-1} 2^{j} =
(2^l - 1) - (2^{l-1} - 1) = 2^l - 2^{l-1} = 2^{l-1}
\end{equation}
\label{eqn:compose}

\subsubsection{Composing Functions with Single-Bit Output}

First we will show how to compose a function table with a single-bit output,
since this is the simplest case. However, $f$ and $g$ may be multi-bit output
functions ($k_s, k_o > 1$). To solve this, we just decompose $f$ and $g$
into a bunch of single-bit functions, and then show how to combine these
functions so that we end up composing the original, multi-bit functions.

For the single-bit case, let's concentrate on $f$ which has $k_s = 1$
and $2^{k_s} = 2$ lines.
(the treatment of $g$ is equivalent with $k_o = 1$ and also $2^{k_o} = 2$ lines).

This is the circuit for composing two such functions, $F_{k,i}$ and $F_{k,i+1}$,
into a new function $F_{k+1,\lfloor 1 \rfloor}$
which is equivalent to first applying
$F_{k,i}$ to get an output and feeding that into $F_{k,i+1}$. In circuit below,
the ``input'' is actually the preceding state $s_i$.

\begin{displaymath}
\Qcircuit @C=2em @R=1.5em {
F_{k,i}^j  & & s_i = 0 & & \qw & \ctrl{3} & \qw      & \ctrlo{2} & \qw       & \qw \\
           & & s_i = 1 & & \qw & \qw      & \ctrl{2} & \qw       & \ctrlo{1} & \qw \\
F_{k,i+1}^j & & s_i = 0 & & \qw & \qw      & \qw      & \ctrl{2}  & \ctrl{3}  & \qw \\
           & & s_i = 1 & & \qw & \ctrl{1} & \ctrl{2} & \qw       & \qw       & \qw \\
F_{k+1,i/2} & & s_i = 0 & & \qw & \targ    & \qw      & \targ     & \qw       & \qw \\
 & & s_i = 1 & & \qw & \qw      & \targ    & \qw       & \targ     & \qw \\
}
\end{displaymath}

\subsubsection{Composing Functions with Multiple-Bit Output}

Now assuming that $k_s > 1$ and we have decomposed such a function table into
$k_s$ single-bit output functions, how do we combine them together again?
We consider the case for $k_s = 2$, shown below, and it is easy to generalize
this for higher $k_s$.
Each $s_i$ is a 2-bit state
where $s_{i,0}$ is the first bit and $s_{i,1}$ is the second bit.
Each $F_i^j$ is a single-bit output function that supplies bit $j$
from iteration $i$ to feed into the next iteration $i+1$.

\xymatrix{
          & *+[F]{F_{i,0}} \ar[r] & s_{i+1,0} \ar[r] \ar[ddr] & *+[F]{F_{i+1,0}} \ar[r] & s_{i+2,0} \\
s_i \ar[ur] \ar[dr] &              &                     &           \\
          & *+[F]{F_{i,1}} \ar[r] & s_{i+1,1} \ar[r] \ar[uur] & *+[F]{F_{i+1,1}} \ar[r] & s_{i+2,1} \\
}

The circuit for composing three of these functions is given below, for
the input state $s_i$ equal to $00$ and $01$. Similar gates can be found
for the remaining state values of $10$ and $11$.
$F_i^0$ and $F_i^1$ supply two bits which become the input for
$F_{i+1}^0$. When composed into the new function $F_{i+1}^0$,
we have a table which produces the first bit of the state $s_{i+2}$
give all the bits of state $s_{i}$.

\begin{displaymath}
\Qcircuit @C=2em @R=1.5em {
F_{k,i}^0     & & s_i = 00 & & \qw & \ctrl{4} & \ctrlo{4}   & \ctrl{4} & \ctrlo{4}  & \qw      & \qw         & \qw       & \qw       & \qw \\
              & & s_i = 01 & & \qw & \qw      & \qw         & \qw       & \qw       & \ctrl{4} & \ctrlo{4}   & \ctrl{4} & \ctrlo{4}  & \qw \\
              & & s_i = 10 & & \qw & \qw      & \qw         & \qw       & \qw       & \qw      & \qw         & \qw       & \qw       & \qw \\
              & & s_i = 11 & & \qw & \qw      & \qw         & \qw       & \qw       & \qw      & \qw         & \qw       & \qw       & \qw \\
F_{k,i}^1     & & s_i = 00 & & \qw & \ctrl{7} & \ctrl{5}    & \ctrlo{6} & \ctrlo{4} & \qw      & \qw         & \qw       & \qw       & \qw \\
              & & s_i = 01 & & \qw & \qw      & \qw         & \qw       & \qw       & \ctrl{6} & \ctrl{4}    & \ctrlo{5} & \ctrlo{3} & \qw \\
              & & s_i = 10 & & \qw & \qw      & \qw         & \qw       & \qw       & \qw      & \qw         & \qw       & \qw       & \qw \\
              & & s_i = 11 & & \qw & \qw      & \qw         & \qw       & \qw       & \qw      & \qw         & \qw       & \qw       & \qw \\
F_{k,i+1}^0   & & s_i = 00 & & \qw & \qw      & \qw         & \qw       & \ctrl{4}  & \qw      & \qw         & \qw       & \ctrl{5}  & \qw \\
              & & s_i = 01 & & \qw & \qw      & \ctrl{3}    & \qw       & \qw       & \qw      & \ctrl{4}    & \qw       & \qw       & \qw \\
              & & s_i = 10 & & \qw & \qw      & \qw         & \ctrl{2}  & \qw       & \qw      & \qw         & \ctrl{3}  & \qw       & \qw \\
              & & s_i = 11 & & \qw & \ctrl{1} & \qw         & \qw       & \qw       & \ctrl{2} & \qw         & \qw       & \qw       & \qw \\
F_{k+1,i/2}^0 & & s_i = 00 & & \qw & \targ    & \targ       & \targ     & \targ     & \qw      & \qw         & \qw       & \qw       & \qw \\
              & & s_i = 01 & & \qw & \qw      & \qw         & \qw       & \qw       & \targ    & \targ       & \targ     & \targ     & \qw \\
              & & s_i = 10 & & \qw & \qw      & \qw         & \qw       & \qw       & \qw      & \qw         & \qw       & \qw       & \qw \\
              & & s_i = 11 & & \qw & \qw      & \qw         & \qw       & \qw       & \qw      & \qw         & \qw       & \qw       & \qw \\
}
\end{displaymath}

To compute each bit of the next state
$s_{i+1}$, a multiple-bit composition consists of 
$k_s$ single-bit compositions. Each single-bit composition is
composed of $2^{k_s}$ generalized Toffoli's with $k_s +1$ controls, and
$2^{k_s}$ NOT gates for negative controls. Remember that a generalized
Toffoli with $k_s + 1$ controls requires $k_s$ Toffoli's and $k_s - 1$
ancillae. Each single-bit composition requires $2^{k_s}$ ancillae to
hold the new function table. The single-bit compositions cannot be done
in parallel, however, because they all use the same control bits.

From the beginning of this section, we have that there are $2^l - 1$
multiple-bit compositions, in general, which multiplies everything above,
except the circuit depth. The circuit depth multiplier is
just the height of the tree of compositions is $l$.

\begin{table}[!h]
\begin{center}
\begin{tabular}{|r|cc|}
\hline
NOT Gates per Multiple-Bit Composition & $\ell_N = $ & $k_s \cdot 2^{k_s} \cdot 2$ \\
Toffoli Gates per Multiple-Bit Composition & $\ell_T = $ & $k_s \cdot (2^{k_s}\cdot (k_s-1)) \cdot 2$\\
Total NOT Gates & & $(2^{l-1}) \cdot \ell_N$ \\
Total Toffoli Gates & & $(2^{l-1}) \cdot \ell_T$ \\
Ancillae & & $(2^l - 1)\cdot k_s \cdot (2^{k_s} + (k_s - 1)) \cdot 2$\\
Circuit Depth & & $l\cdot(\ell_T d_T + \ell_N d)$\\
\hline
\end{tabular}
\caption{Resource counts for Step 3: composing narrowed function tables in parallel.}
\end{center}
\end{table}

\subsection{Applying Function Tables}
\label{subsec:apply}

Finally, we have narrowed tables composed so that we can skip ahead to
intermediate states in a logarithmic-depth, binary tree fashion.

The scheme for determining all the intermediate states $s_i$ follows
the diagram below. Recall that we start out knowing the initial state
$s_0$.

\begin{displaymath}
\xymatrix @C=1em {
\text{level 1} & s_0 \ar[rrrr]^{F_{2,0}} \ar[ddd] &     &        &     & s_4 \ar[ddd]\\
\text{level 2} &     \ar[rr]^{F_{1,0}}   &     & s_2 \ar[dd]    &     & \\
\text{level 3} &     \ar[r]^{F_{0,1}} & s_1 \ar[d] & \ar[r]^{F_{0,3}} & s_3 \ar[d] & \\
              &               &     &        &     &
}
\end{displaymath}

In general, each composed function has multiple-bit output, and to
apply it we actually need to apply $k_s$ separate single-bit output
functions. Applying each single-bit output function takes $2^{k_s}$
generalized Toffoli's,
each with $k_s+1$ controls (one for every bit of the input, plus one for
the function table to apply) and in total requiring $2^{k_s}$ negative
controls. Recall that such a generalized Toffoli can be implemented
with $k_s - 1$ normal Toffoli's and $k_s - 2$ ancillae.
Finally, to hold the final value of the $s_i$'s, we need $k_s$ ancillae.

\begin{displaymath}
\Qcircuit @C=2em @R=1.5em {
F_{k,i}^j & & s_i = 00 & & \qw & \ctrl{4} & \qw      & \qw      & \qw      & \qw\\
        & & s_i = 01 & & \qw & \qw       & \ctrl{3}  & \qw       & \qw      & \qw\\
        & & s_i = 10 & & \qw & \qw       & \qw       & \ctrl{2}  & \qw      & \qw\\
        & & s_i = 11 & & \qw & \qw       & \qw       & \qw       & \ctrl{1} & \qw\\
s_i^0   & &          & & \qw & \ctrlo{1} & \ctrlo{1} & \ctrl{1}  & \ctrl{1} & \qw \\
s_i^1   & &          & & \qw & \ctrlo{1} & \ctrl{1}  & \ctrlo{1} & \ctrl{1} & \qw \\
s_{i+1}^j & &        & & \qw & \targ     & \targ     & \targ     & \targ    & \qw \\
}
\end{displaymath}

Unlike in the previous step with compositions, we need to apply the
lowest level of our function tables, the $F_{0,i}$ functions. Therefore,
in analogy to Equation \ref{eqn:compose},
the total number of function applications is given below. The
circuit depth depends on $l$ levels of applications, and not $l+1$
because the two highest levels (to compute $s_n$ and $s_{n/2}$) can be
done in parallel.

\begin{equation}
\left( \sum_{j=0}^{l} 2^j \right) - \left( \sum_{j=1}^{l} 2^j \right)
= (2^{l+1} - 1) - (2^{l} - 1) = 2^l = n
\end{equation}

Also unlike with compositions, we bring back our optional output function
$g$, which we have already narrowed and is ready to be applied at each
of $n$
iteration with the states $s_i$ which we have just computed in this step.
Applying $g$ is the same as applying $f$ (and $F_{k,i}$) above.

\begin{table}
\begin{center}
\begin{tabular}{|r|cc|}
\hline
NOT Gates per Application & $\ell_N = $ & $2^{k_s} \cdot 2$ \\
Toffoli Gates per Application & $\ell_T = $ & $k_s \cdot (2^{k_s}\cdot (k_s-1)) \cdot 2$\\
Total NOT Gates & & $n \cdot (1 + \delta_g) \cdot \ell_N$ \\
Total Toffoli Gates & & $n \cdot (1 + \delta_g) \cdot \ell_T$ \\
Ancillae & & $(2^l)\cdot k_s \cdot (k_s + (k_s - 2)) \cdot 2$\\
Circuit Depth & & $l\cdot(\ell_T d_T + \ell_N d)$\\
\hline
\end{tabular}
\end{center}
\end{table}

