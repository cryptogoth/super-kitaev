\subsection{Composing Narrowed Function Tables}
\label{subsec:compose}

Now that we have narrowed the function tables by input, they only depend
on the current state. By composing adjacent function tables in a binary
tree hierarchy, we can compute all the intermediate states of an FA
execution while still preserving logarithmic depth.

We will use the notation $F_{k,i}$ to denote the $i$th composed function table
at the $k$th level of recursion, according to the diagram below for $n=4$.
$F_{0,i}$, at the base level, corresponds to the original function $f$
narrowed by input $a_i$.

\begin{displaymath}
\xymatrix @C=1em {
  \ar[rr]^{s_0} & &
   *+[F]{F_{0,0}} \ar[rr]^{s_1} & &
   *+[F.]{F_{0,1}} \ar[rr]^{s_2} & &
   *+[F]{F_{0,2}} \ar[rr]^{s_3} & &
   *+[F.]{F_{0,3}} \ar[rr]^{s_4} & & \\
  & & & *+[F]{F_{1,0}} \ar@{<-}[ul] \ar@{<-}[ur]
  & & & & *+[F.]{F_{1,1}} \ar@{<-}[ul] \ar@{<-}[ur] \\
  & & & & & *+[F]{F_{2,0}} \ar@{<-}[ull] \ar@{<-}[urr]
}
\end{displaymath}

How many composition operations do we need to do for our $n$ initial
narrowed $F_{0,i}$ tables? The resulting full binary tree (except for
the first level) has $\lceil \log_2 n \rceil = l$ levels and therefore
$2^{l}-1$ nodes. However, as will become apparent in the next section, we don't
need the compositions with a dotted border above.

Since at each level of compositions, we can remove half of the nodes
(except for the root node), we have the following formula of the
total number of compositions we need for $n$ iterations and $l$ levels,
where the root level corresponds to $j=0$.

\begin{equation}
\sum_{j=0}^{l-1} 2^j - \sum_{j=1}^{l-1} 2^{j} =
(2^l - 1) - (2^{l-1} - 1) = 2^l - 2^{l-1} = 2^{l-1}
\end{equation}
\label{eqn:compose}

\subsubsection{Composing Functions with Single-Bit Output}

First we will show how to compose a function table with a single-bit output,
since this is the simplest case. However, $f$ and $g$ may be multi-bit output
functions ($k_s, k_o > 1$). To solve this, we just decompose $f$ and $g$
into a bunch of single-bit functions, and then show how to combine these
functions so that we end up composing the original, multi-bit functions.

For the single-bit case, let's concentrate on $f$ which has $k_s = 1$
and $2^{k_s} = 2$ lines.
(the treatment of $g$ is equivalent with $k_o = 1$ and also $2^{k_o} = 2$ lines).

This is the circuit for composing two such functions, $F_{k,i}$ and $F_{k,i+1}$,
into a new function $F_{k+1,\lfloor 1 \rfloor}$
which is equivalent to first applying
$F_{k,i}$ to get an output and feeding that into $F_{k,i+1}$. In circuit below,
the ``input'' is actually the preceding state $s_i$.

\begin{displaymath}
\Qcircuit @C=2em @R=1.5em {
F_{k,i}^j  & & s_i = 0 & & \qw & \ctrl{3} & \qw      & \ctrlo{2} & \qw       & \qw \\
           & & s_i = 1 & & \qw & \qw      & \ctrl{2} & \qw       & \ctrlo{1} & \qw \\
F_{k,i+1}^j & & s_i = 0 & & \qw & \qw      & \qw      & \ctrl{2}  & \ctrl{3}  & \qw \\
           & & s_i = 1 & & \qw & \ctrl{1} & \ctrl{2} & \qw       & \qw       & \qw \\
F_{k+1,i/2} & & s_i = 0 & & \qw & \targ    & \qw      & \targ     & \qw       & \qw \\
 & & s_i = 1 & & \qw & \qw      & \targ    & \qw       & \targ     & \qw \\
}
\end{displaymath}

\subsubsection{Composing Functions with Multiple-Bit Output}

Now assuming that $k_s > 1$ and we have decomposed such a function table into
$k_s$ single-bit output functions, how do we combine them together again?
We consider the case for $k_s = 2$, shown below, and it is easy to generalize
this for higher $k_s$.
Each $s_i$ is a 2-bit state
where $s_{i,0}$ is the first bit and $s_{i,1}$ is the second bit.
Each $F_i^j$ is a single-bit output function that supplies bit $j$
from iteration $i$ to feed into the next iteration $i+1$.

\xymatrix{
          & *+[F]{F_{i,0}} \ar[r] & s_{i+1,0} \ar[r] \ar[ddr] & *+[F]{F_{i+1,0}} \ar[r] & s_{i+2,0} \\
s_i \ar[ur] \ar[dr] &              &                     &           \\
          & *+[F]{F_{i,1}} \ar[r] & s_{i+1,1} \ar[r] \ar[uur] & *+[F]{F_{i+1,1}} \ar[r] & s_{i+2,1} \\
}

The circuit for composing three of these functions is given below, for
the input state $s_i$ equal to $00$ and $01$. Similar gates can be found
for the remaining state values of $10$ and $11$.
$F_i^0$ and $F_i^1$ supply two bits which become the input for
$F_{i+1}^0$. When composed into the new function $F_{i+1}^0$,
we have a table which produces the first bit of the state $s_{i+2}$
give all the bits of state $s_{i}$.

\begin{displaymath}
\Qcircuit @C=2em @R=1.5em {
F_{k,i}^0     & & s_i = 00 & & \qw & \ctrl{4} & \ctrlo{4}   & \ctrl{4} & \ctrlo{4}  & \qw      & \qw         & \qw       & \qw       & \qw \\
              & & s_i = 01 & & \qw & \qw      & \qw         & \qw       & \qw       & \ctrl{4} & \ctrlo{4}   & \ctrl{4} & \ctrlo{4}  & \qw \\
              & & s_i = 10 & & \qw & \qw      & \qw         & \qw       & \qw       & \qw      & \qw         & \qw       & \qw       & \qw \\
              & & s_i = 11 & & \qw & \qw      & \qw         & \qw       & \qw       & \qw      & \qw         & \qw       & \qw       & \qw \\
F_{k,i}^1     & & s_i = 00 & & \qw & \ctrl{7} & \ctrl{5}    & \ctrlo{6} & \ctrlo{4} & \qw      & \qw         & \qw       & \qw       & \qw \\
              & & s_i = 01 & & \qw & \qw      & \qw         & \qw       & \qw       & \ctrl{6} & \ctrl{4}    & \ctrlo{5} & \ctrlo{3} & \qw \\
              & & s_i = 10 & & \qw & \qw      & \qw         & \qw       & \qw       & \qw      & \qw         & \qw       & \qw       & \qw \\
              & & s_i = 11 & & \qw & \qw      & \qw         & \qw       & \qw       & \qw      & \qw         & \qw       & \qw       & \qw \\
F_{k,i+1}^0   & & s_i = 00 & & \qw & \qw      & \qw         & \qw       & \ctrl{4}  & \qw      & \qw         & \qw       & \ctrl{5}  & \qw \\
              & & s_i = 01 & & \qw & \qw      & \ctrl{3}    & \qw       & \qw       & \qw      & \ctrl{4}    & \qw       & \qw       & \qw \\
              & & s_i = 10 & & \qw & \qw      & \qw         & \ctrl{2}  & \qw       & \qw      & \qw         & \ctrl{3}  & \qw       & \qw \\
              & & s_i = 11 & & \qw & \ctrl{1} & \qw         & \qw       & \qw       & \ctrl{2} & \qw         & \qw       & \qw       & \qw \\
F_{k+1,i/2}^0 & & s_i = 00 & & \qw & \targ    & \targ       & \targ     & \targ     & \qw      & \qw         & \qw       & \qw       & \qw \\
              & & s_i = 01 & & \qw & \qw      & \qw         & \qw       & \qw       & \targ    & \targ       & \targ     & \targ     & \qw \\
              & & s_i = 10 & & \qw & \qw      & \qw         & \qw       & \qw       & \qw      & \qw         & \qw       & \qw       & \qw \\
              & & s_i = 11 & & \qw & \qw      & \qw         & \qw       & \qw       & \qw      & \qw         & \qw       & \qw       & \qw \\
}
\end{displaymath}

To compute each bit of the next state
$s_{i+1}$, a multiple-bit composition consists of 
$k_s$ single-bit compositions. Each single-bit composition is
composed of $2^{k_s}$ generalized Toffoli's with $k_s +1$ controls, and
$2^{k_s}$ NOT gates for negative controls. Remember that a generalized
Toffoli with $k_s + 1$ controls requires $k_s$ Toffoli's and $k_s - 1$
ancillae. Each single-bit composition requires $2^{k_s}$ ancillae to
hold the new function table. The single-bit compositions cannot be done
in parallel, however, because they all use the same control bits.

From the beginning of this section, we have that there are $2^l - 1$
multiple-bit compositions, in general, which multiplies everything above,
except the circuit depth. The circuit depth multiplier is
just the height of the tree of compositions is $l$.

\begin{table}[!h]
\begin{center}
\begin{tabular}{|r|cc|}
\hline
NOT Gates per Multiple-Bit Composition & $\ell_N = $ & $k_s \cdot 2^{k_s} \cdot 2$ \\
Toffoli Gates per Multiple-Bit Composition & $\ell_T = $ & $k_s \cdot (2^{k_s}\cdot (k_s-1)) \cdot 2$\\
Total NOT Gates & & $(2^{l-1}) \cdot \ell_N$ \\
Total Toffoli Gates & & $(2^{l-1}) \cdot \ell_T$ \\
Ancillae & & $(2^l - 1)\cdot k_s \cdot (2^{k_s} + (k_s - 1)) \cdot 2$\\
Circuit Depth & & $l\cdot(\ell_T d_T + \ell_N d)$\\
\hline
\end{tabular}
\caption{Resource counts for Step 3: composing narrowed function tables in parallel.}
\end{center}
\end{table}