\section{The Main Algorithm}
\label{sec:main-algo}

The Super-Kitaev algorithm takes a radically different approach than
Solovay-Kitaev by identifying that most gates can be easily reduced to
$\mathcal{G}$ in constant time with the exception of the ``targetless''
controlled phase operator, which will take some sweat to approximate.

\begin{displaymath}
\Lambda(e^{i\phi}) = 
 \left(
  \begin{array}{cc}
    1 & 0 \\
    0 & e^{i\phi} \\
  \end{array} \right)
\end{displaymath}

Thankfully \cite{ksv02} is a self-contained reference
for enacting $\Lambda(e^{i\phi})$ with several modules which are useful in
their own right:
parallelized phase estimation procedure, parallel iteration of finite
automata, and a logarithmic depth quantum arithmetic operations such
as addition (and subtraction) described in \ref{subsec:add},
multiplication described in \ref{subsec:mult}, and division described
in \ref{subsec:div}.

It is now useful to discuss certain states with magical properties.
Their superposition is easy to form, they are the eigenstates of the
afore-mentioned
addition operator. Used with phase estimation, this would allow us to enact
a random phase shift simply by adding numbers, which is better than not
being able to shift phases at all (or only being able to do it with a
complicated operator). Moreover, it turns out we can copy these states
also using simple addition. These ideas make Super-Kitaev a beautiful result
regardless of the outcome of our later numerical comparisons.

\subsection{Magic States}
\label{subsec:magic-state}

Suppose you had some magic state $\ket{\psi}$, given to you from some
all-powerful being, parameterized by the
integers $n$ and $k$ such that

\begin{equation*}
\ket{\psi_{n,k}} = \frac{1}{\sqrt{2^n}} \sum_{j=0}^{2^n-1}
e^{-2\pi i j k / 2^n} \ket{j}
\end{equation*}

where $0 < k < 2^n$, $k$ is odd.

Note that the magic states $\ket{\psi}$ look suspiciously like a
QFT state of the computational basis.
However that doesn't help us here, because QFT in general requires
$O(n^2)$-depth and we need a logarithmic-depth compiler.
This is intuitively why these states are hard to create.
If we were able to construct them easily, we would also have a quicker way
of implementing QFT.

However, what we can do is create a superposition over all odd $k=(2s-1)$
by starting in the state $\ket{0}^{\otimes n}$,
applying a Hadamard to the most significant qubit to get an equal mix of $\ket{0}$
and $\ket{1}$, then put a negative sign on the $\ket{1}$ component by applying
$\sigma^z$ to the same qubit.

\begin{equation*}
\ket{\eta} = \normtwo \ket{0} - \normtwo \ket{2^{n-1}} =
\frac{1}{\sqrt{2^{n-1}}} \sum_{s=1}^{2^{n-1}} \ket{\psi_{n,2s-1}}
\end{equation*}

More obviously relevant to our overall goal of approximating
$\Lambda(e^{i\phi})$, we can enact a phase
shift simply by doing addition, which is a result of these states
being eigenvectors of the modular addition operator defined
%by $A : \ket{j} \rightarrow \ket{(j+1) mod 2^n}$
 
\begin{equation*}
A\ket{\psi_{n,k}} = e^{2\pi i \phi_k} \ket{\psi_{n,k}}
\end{equation*}

\begin{equation*}
A\ket{j} \rightarrow \ket{j+1 \mod 2^n}
\end{equation*}

where finding the eigenvalue $e^{2\pi i \phi_k}$ corresponds to finding
the phase $\phi_k = k / 2^n$.

Repeated application of $A$ (say $p$ times) would result in a phase
added to the eigenstate equal to a multiple of $e^{2\pi i p / 2^n}$

\begin{equation}
A^p\ket{\psi_{n,k}} = e^{2\pi i \phi_k / 2^n} \ket{\psi_{n,k}}
\end{equation}

This explains why we don't find even $k$ interesting,
since then we would not get a
cyclic distribution of $2^n$ different phases,
since only odd $k$
are coprime with $2^n$. The exception is $k=0$, since this is the
equal superposition of computational basis states, which we can also
efficiently create. This will be a useful starting point later on to
create magic states
for odd $k$.

\begin{displaymath}
\ket{\psi_{n,0}} = H^{\otimes n}\ket{0^n}
\end{displaymath}

Suppose we have a certain state $\psi_{n,k}$ but we want to get enact
a phase shift $e^{2\pi i l / 2^n}$. We can do this by solving $p=p(s,l)$
in this equation:

\begin{equation}
\label{eqn:psl}
(2s-1)p \equiv l (\mod 2^n)
\end{equation}

Stipulating $k$ to be odd guarantees that there is a unique solution $p$.

We then applying $A^p$ as follows:

\begin{equation}
\label{eqn:upsilon}
\Upsilon_n(A) \ket{p, \psi_{n,k}} \rightarrow
e^{2\pi i l/2^n} \ket{p, \psi_{n,k}}
\end{equation}

But $p$ is in general hard to solve for, unless $k=1$, in which case
$p = l$. Therefore, we will later see that $\ket{\psi_{n,1}}$ is a desirable
state to have.

Finally, to copy the state $\ket{\psi_{n,k}}$ it suffices to apply the following
operator which only uses subtraction (addition with one addend and the
outcome negated in two's complement representation).

\begin{equation*}
\ket{\psi_{n,k}}^{\otimes m} = W^{-1}\left( \ket{\psi_{n,0}}^\otimes(m-1) \otimes \ket{\psi_{n,k}} \right)
\end{equation*}

where $W$ is defined by

\begin{multline}
W : \ket{x_1,\ldots,x_{m-1},x_m} \rightarrow \\
 \ket{x_1,\ldots,x_{m-1},x_1+\ldots+x_m}
\end{multline}

Armed with these properties, we're now ready to enact arbitrary phase shifts.

%For the steps below, we note that we can avoid solving the equation $kp \equiv l (mod 2^n)$
%if we set $k=1$. This is equivalent to realizing $\Upsilon_n(e^{2\pi i / 2^n})$
%by applying $\Upsilon(A)$ from Section \ref{subsec:phase-shift} to the state
%$\ket{\psi_{n,1}}$. So we need to create one copy of this magic state to
%simulate each of $m$ $\Lambda(e^{i\phi})$ gates, $m \le L'$.


\subsection{Super-Kitaev Steps}

Given a circuit $C$ to compile,

\begin{enumerate}
\item Precompile $C$ into gates from $\mathcal{G} \cup \{\Lambda(e^{2\pi i l / 2^n})\}$
using the results from Section \ref{subsec:precompile} in $O(1)$ time, depth,
and size.
Now we are done with the single-qubit gates and CNOT, and we have computed
the values $\{l_1, \ldots , l_m\}$ that allow us to approximate our
desired $m$
$\Lambda(e^{i\phi})$ gates as $\phi \approx l/2^n$ to within precision
$2^{-n}$.
\item Create the state magic state $\ket{\psi_{n,0}}$ with $n$ Hadamards.
\item Turn it into $\ket{\psi_{n,1}} = \Upsilon(e^{-2\pi i / 2^n}) \ket{\psi_{n,0}}$
using the registered phase shift procedure in Section \ref{subsec:phase-shift}
This is done with a circuit of size $O(n^2\log n)$ and $O((\log n)^2)$ depth.
\item Make $m$ copies of the state $\ket{\psi_{n,1}}$ out of one copy by 
applying the addition operation $A$.
\item Simulate each $\Lambda(e^{2\pi i l / 2^n})$
using one copy each of $\ket{\psi_{n,1}}$, to which we can add our
values $l$ using $\Upsilon(A)$.
This takes size $O(mn)$ and depth $O(\log n)$, since we can enact
all these phase shifts in parallel.
\end{enumerate}

Now for the resource calculations of these individual steps and their
substeps.

\subsection{Precompiling to the Standard Set}
\label{subsec:precompile}

In order to perform the precompilation step, we need to know that we can
change any unitary in $SU(d)$ by tensor products of unitaries in $SU(2)$
and $SU(4)$, that is, single- and two-qubit gates.

To prove the universality of the standard set $\mathcal{Q}$ is beyond the
scope of these notes. For our purposes, we assume that we can precompile
any gate down to single-qubit rotations $U \in SU(2)$ and
controlled-single-qubit rotations $\Lambda(U), U\in SU(2)$.
(For example, it's known how to do that for the Toffoli gate, which is what we
need to implement our quantum adder circuit).
From there, how do we get down to the standard set?

%\begin{theorem}[Lemma 8.2]
%\label{lemma82}
%Any arbitrary unitary operator $U$ of dimension $d \times d$ can be represented
%as a product of $d(d-2)/2$ matrices of the form:
%
%\begin{equation*}
%\left( \begin{array}{cccccccc}
%     1 & 0      & \cdots & \cdots & \cdots & \cdots & \cdots & \cdots\\
%     0 & 1      & \cdots & \cdots & \cdots & \cdots & \cdots & \cdots\\
%\vdots & \cdots & \ddots & \cdots & \cdots & \cdots & \cdots & \cdots\\
%\vdots & \cdots & \cdots &      a &      b & \cdots & \cdots & \cdots\\
%\vdots & \cdots & \cdots &      c &      d & \cdots & \cdots & \cdots\\
%\vdots & \cdots & \cdots & \cdots & \cdots & \vdots & \cdots & \cdots\\
%\vdots & \cdots & \cdots & \cdots & \cdots & \cdots & 1 & 0\\
%\vdots & \cdots & \cdots & \cdots & \cdots & \cdots & 0 & 1\\
%\end{array}
%\right)
%\end{equation*}
%
%where $\left( \begin{array}{cc}
%a & b\\
%c & d\\
%\end{array} \right) \in SU(2)$
%\end{theorem}
%
%\begin{proof}
%The running time of this algorithm is $O(M^3) \cdot \mathrm{poly}(\log(1/\delta)$.
%\end{proof}

\begin{theorem}[Problem 8.1]
Any single-qubit gate $U$ can be simulated using $\mathcal{Q} \cup \{ \Lambda(e^{i\phi}) \}$.
\end{theorem}

\begin{proof}
Using the Euler angle decomposition into $X$ and $Z$ rotations, we can express any
unitary $U$ as follows, with real angles $\phi$, $\gamma$, $\beta$, $\alpha$:

\begin{equation}
U = e^{i\phi}e^{i(\gamma/2)\sigma^z}e^{i(\beta/2)\sigma^x}e^{i(\alpha/2)\sigma^z}
\end{equation}
This involves solving four equations in four variables and can be done in constant time.

Now how do we implement each of the elements of the Euler angle decomposition from
the $\mathcal{Q}$? Using these identities:

\begin{equation}
e^{i\phi} = \Lambda(e^{i\phi})\sigma^x\Lambda(e^{i\phi})\sigma^x
\end{equation}
\begin{equation}
\sigma^x = H\Lambda(e^{i\pi})H
\end{equation}
\begin{equation}
e^{i\phi\sigma^z} = \Lambda(e^{-i\phi}\sigma^x\Lambda(e^{i\phi})\sigma^x
\end{equation}
\begin{equation}
e^{i\phi\sigma^x} = H e^{i\phi\sigma^z} H
\end{equation}
\end{proof}

\begin{theorem}[Problem 8.2]
Any controlled operator $\Lambda(U)$ where $U \in SU(2)$ can be simulated using
$\mathcal{Q} \cup \{ \Lambda(e^{i\phi})$.
\end{theorem}

\begin{proof}
We can represent the controlled operator as $\Lambda(U) = \Lambda(e^{i\phi})\Lambda(V)$.
$\Lambda(e^{i\phi})$ is already in our set, so it remains to implement
$\Lambda(V)$, where $V \in SU(2)$. Geometrically, we can decompose any
three-dimensional rotation into two $180^{\circ}$ rotations about appropriate
axes. Using the homomorphism between $SO(3)$ and $SU(2)$ and the fact that
$\sigma^x$ corresponds to a $180^{\circ}$ rotation about the $x$-axis.

\begin{equation}
V = A\sigma^x A^{-1} B\sigma^x B^{-1}
\end{equation}
\end{proof}


\subsection{Registered-Phase Shift}
\label{subsec:phase-shift}

Registered phase shifting is the only use of phase estimation in
Super-Kitaev but it is the most resource-intensive step.
We would like to enact the operator $\Upsilon(e^{-2\pi i / 2^n})$ on
$\ket{\psi_{n,0}}$, which is easy to create, to get $\ket{\psi_{n,1}}$,
which we can later use to enact $\Lambda(e^{2\pi i l / 2^n})$.

\begin{enumerate}
\item Create the state $\ket{\eta}$ as described in \ref{subsec:magic-state}.
\item Measure $k$ by determining the phase $\phi_k = k/2^n$ with precision
$\delta = 2^{-n}$ by using phase estimation. One of the $n$-qubit registers
now contains $\ket{\psi_{n,k}}$ for some fixed $k$.
This is done by parallelized phase estimation described in
\ref{subsec:ppe}.
This is done by a circuit of size $O(n^2)$ and depth $O(\log n)$.
\item Solve for $p$ in equation
Equation \ref{eqn:psl} using modular division, described in
\ref{subsec:div}.
\item Apply $A^p$ to the $n$-qubit register containing $\ket{\psi}$, which
enacts the desired phase shift.
\end{enumerate}


\subsection{Parallelized Phase Estimation}
\label{subsec:ppe}

One of the key components of the registered phase shifting procedure
described in the previous section is the ability to ``pick'' a random
eigenstate $\ket{\psi_k}$ of a unitary operator $U$ and
measure its corresponding eigenvalue (phase) $\phi_k$
with some degree of precision $\delta = 2^{-n}$ and
error probability $\epsilon = 2^{-l}$. As $n$ increases, the phases
generally become closer together, which is why we need exponential precision
to distinguish between them.
Of course, this exactly describes the phase
estimation procedure, a key technique in many quantum algorithms developed
by Kitaev in his derivation of Shor's factoring result \cite{kitaev}.

\begin{displaymath}
U\ket{\psi_k} = e^{2\pi i \phi_k} \ket{\psi_k}
\end{displaymath}

Phase estimation holds some superposition of eigenstates
$\sum_{i} \alpha_i \ket{\psi_i}$
in an $n$-qubit target register, to which it applies repeated measuring
operators $\Lambda(U^{2^k})$
controlled on some $t$-qubit register, which holds an
approximation $\tilde{phi}$ to the real phase $\phi$.
The unitary $U$ is applied in successive powers of two to get
power-of-two multiples of the phase for increased precision.
The error probability of approximating the phase to within a given
precision is given by the following:

\begin{displaymath}
\Pr\left[ | \phi - \tilde{phi} | \ge \delta \right] \le \epsilon
\end{displaymath}

The parameter $t = t(\delta, \epsilon)$ encodes the dependence of the number
of $\Lambda(U)$ measuring operators as a function of our desired
$\delta$ and $\epsilon$.
It varies according to the exact phase estimation procedure
used.

The popular version of phase estimation presented in \cite[nc00],
requires $t$ repeated controlled applications of some unitary
$U$ (and its successive powers as $U^{2^k}$, $0<k<2^t$)
to a target state which holds some superposition of its eigenvectors,
controlled by $t$ bits which will hold the approximation to a corresponding
eigenvalue (phase).
This version requires applying an inverse quantum Fourier
transform (QFT), which is already high-depth and way more inefficient than our
desired quantum compiler.

To achieve our desired low-depth, we can ``parallelize'' the application of
$\Lambda(U)$ by interpreting the
$t = (n+2)s$ control bits as an $n$-bit number $q$ and
apply $\Lambda(A^q)$ only once.
In the Super-Kitaev procedure, $A$ is the addition operator on an $n$ qubit
target register containing $\psi_{n,k}$, so we can
only effectively add the lowest $n$ bits of $q$.
Furthermore, the eigenvalues of $A$ are rational with a fixed
denominator, $\phi_k = k / 2^n$.
To avoid the inverse QFT,
we can do a classical postprocessing step, which we'll mostly skip over
for fairly good reasons, and then we'll do a detailed resource count of
parallelized phase estimation.

\subsection{Classical Postprocessing}

It is now the point to mention that Kitaev's phase estimation procedure
contains a post processing step which is completely classical in
character, in that they involve a measurement. If this measurement is
projective and the outcomes are completely classical, the remaining steps
can be done on our classical computer (recall our quantum coprocessor model),
and the results fed back into our quantum subroutine, registered
phase shifting. Therefore, as long as we can perform these classical
algorithms in polynomial time (which we can), we don't really care
about the equivalent circuit size and depth.

The steps of classical postprocessing, which will determine some of the
parameters in the earlier, quantum part of phase estimation are as follows.

\begin{enumerate}

\item
Estimate the phase and its power-of-two multiples
$2^j \phi_k$ to
some constant, modest precision $\delta''$, where
$0 \le j < (n+2)$. For each $j$, we
apply a series of $s$ measuring operators targeting the state $\ket{\nu}$
controlled on $s$ qubits in the state $(\ket{0}+\ket{1})/\sqrt{2}$,
essentially encoding the $2^j \phi_k$ as a bias in a coin, and flipping the
coin $s$ times in a Bernoulli trial, counting the number of $1$ outcomes,
and using that fraction to approximate the real $2^j \phi_k$.
\item
Sharpen our estimate to exponential precision $1/2^(n+2)$ using the
$(n+2)$ estimates, each for different bits in the binary expansion of
$phi_k$. Multiplication by successive powers-of-two shift these bits
up to a fixed position behind the zero in a binary fraction representation,
where we can use a finite-automata and a constant number of
bits to refine our $O(n)$-length running approximation.
\end{enumerate}

Three things are worth mentioning about the interrelation of the parameters
between these two steps. Since our phases all have a denominator of $2^n$,
there is no need to run the continued fractions algorithm on multiple
convergents, as is the case with period-finding in Shor's factoring algorithm.
Furthermore, the phases are $1/2^n$ apart, therefore it suffices to approximate
the phases to within $1/2^{n+2}$ in order to break ties, which is where
our range for $j$ comes from above.

The number of trials $s$ comes from the Chernoff bound:

\begin{displaymath}
\Pr \left[ | s^{-1}\sum_{r=1}^s v_r - p_* | \ge \delta'' \right]
\le 2e^{-2\delta'^2 s}
\end{displaymath}

Setting this equal to the desired error probabiliy $\epsilon$ we get

\begin{displaymath}
s = \frac{1}{2\delta''^2}\ln \frac{1}{\epsilon}
\end{displaymath}

We are actually estimating the values $\cos(2\pi \cdot 2^j \phi_k)$ and
$\sin(2\pi \cdot 2^j \phi_k)$, so if we wish to know $2^j \phi_k$ with
precision $\delta''$, we actually need to determine the $\cos(\cdot)$ and
$\sin(\cdot)$ values with a different precision $\delta'$, lower-bounding
it with the steepest part of the cosine and sine curves.

\begin{displaymath}
\delta' = 1 + cos(\pi - \delta'')
\end{displaymath}

The factor $\frac{1}{2\delta''^2}$ depends on the constant precision with
which we determine our $2^j \phi_k$ values. Since classical time is
cheap and quantum gates are expensive, it makes sense to minimize the number
of trials $s$. The following table shows the corresponding values of $1/(2\delta''^2)$
and $\delta'$ as a function of various choices for $\delta'$.

\begin{center}
\begin{table}
\begin{tabular}{|c|c|c|}
\hline
$\delta''$ & $\delta'$   & $1/(2\delta''2)$\\
\hline
$1/16$     & $0.0019525$ & $131,160$\\
$1/8$      & $0.0078023$ & $  8,213$\\
$1/4$      & $0.0310880$ & $    517$\\
\hline
\end{tabular}
\end{table}
\end{center}

By making our $\delta'$
exponentially worse (doubling it) we are only increasing the range of
$j$ a linear amount (by one). In general, for $\delta'=1/2^l$, we get
a final estimate for $\phi = 2^{m-3}$

However, projective measurements are irreversible, and we may wish to
uncompute the phase estimation procedure to restore our ancilla to
their initialized state and reuse them later on. In that case, the
postprocessing can actually be done on a quantum computer using
reversible gates and without projective measurements. That's why
the authors of \cite{ksv02} go to some care to show that all the classical
postprocessing steps can be done in polynomial-size and logarithmic-depth
circuit.
However, to simplify our analysis, we assume the case
in the previous paragraph, and accept the
loss of $n$ ancilla qubits, After all, we only run phase estimation once
to get our initial $\ket{\psi_{n,1}}$ state.

\subsection{Steps in the Parallelized Phase Estimation Algorithm}

The main steps in parallelized phase estimation as applied to Super-Kitaev
are:

\begin{enumerate}

\item Begin with a $t$-qubit ancilla register initialized to $\ket{0}^{\otimes t}$.
%\textsc{Resources} $= [0,0,0,0,0,t]$

\item Place the $t$-qubit register into an equal superposition by
applying $n$ Hadamard gates.
%\textsc{Resources} $= [0,0,0,n,1,0]$

\item Treat $t$ as $2s$ groups of bits, each encoding an $n$-bit number.
Sum them up out-of-place, retaining only the lowest $n$-bits,
to get the superposition
of all $n$-bit numbers, $1/(\sqrt{2^{n}}) \sum_{i=0}^{2^n-1} \ket{q_i}$.
Call this register $\ket{q_i}$.
%\textsc{Resources} $= ADD-OUT(2s \times n)$

\item Reverse the first step by applying another $n$ Hadamards.
%\textsc{Resources} $= [0,0,0,n,1,0]$

\item Apply the gate $\Upsilon(A)$ to the target $\ket{\nu}$ controlled
on $\ket{q_i}$, which is equivalent to adding all $q_i$ in superposition
(in place).
%\textsc{Resources} $= ADD-IN(2 \times n)$

\item Measure the register $\ket{q_i}$ in the computational basis to get 
a particular value $q$ and collapse the register to $\ket{q}$. All $t=(n-2)s$
now contain classical $0$ or $1$ as outcomes of $(n+2)s$ Bernoulli trials.

\item Read out these outcomes into our classical controller
and perform the postprocessing
described above to get an approximation of $\phi$ with precision $\delta$.

\end{enumerate}


\section{Quantum Adding}

Since adding is such a common
operation in Super-Kitaev and many other quantum algorithms, it's important
for us to make sure we can do these efficiently.
The original development of Super-Kitaev in \cite{ksv02} gives us two
ways to add efficiently.
The first way reduces the sum of three numbers ($a,b,c$) to the sum of
two encoded numbers ($u,v$) in constant depth, simply by encoding the
bitwise sum $a_i+b_i+c_i$ as a 2-bit element from $\{0,1,2,3\}$. We arbitrarily
choose $u_i$ to be the low-order bit and $v_i$ to be high-order.
This can be used to parallelize the sum of $m$ numbers in
a $\log_{3/2}m$-depth tree, with $m-2$ encoded addition operations total.

\begin{figure}
\caption{Kitaev encoded adding: u}
\end{figure}

\begin{figure}
\caption{Kitaev encoded adding: v}
\end{figure}

\begin{figure}
\caption{$\log_{3/2} m$-depth tree for encoded adding of $m$ numbers}
\end{figure}

The encoded adding procedure is out-of-place and requires two ancillae
(one to hold the output bit $u_i$ or $v_i$). In order to apply
Bennett's uncomputing trick to perform this operation in-place, we round
the number of ancillae up to three to match the number of inputs, and we
use 3 SWAP gates (consisting of 3 CNOTs each). Because $v_i$ is high-order,
it always gets shifted up one bit: $v_0 = 0$ and $v_n$ the high-order carry
bit is discarded for sums modulo $2^n$.

The resource counts of Super-Kitaev encoded adding of $m$ $n$-bit numbers
are shown below, both for in-place and out-of-place.

\begin{tabular}{|r|c|c|}
\hline
Resource      & ENC-ADD-IN      & ENC-ADD-OUT\\
Depth         & $9$             & $6$\\ 
Toffoli Gates & $12 (n-1) + 6$  & $6 (n-1) + 6$\\
CNOT Gates    & $21 (n-1) + 15$ & $6 (n-1) + 6$\\
NOT Gates     & $10 (n-1)$      & $5 (n-1)$\\
Ancillae      & $3n$            & $4n/3$\\ 
\hline
\end{tabular}

As usual, the resource counts of the equivalent out-of-place procedure
are smaller.

We can combine the final two encoded numbers
into a normal $(O(\log m) + n)$-bit number using the second way of adding.
This method uses parallelized
iteration of general finite automata (FA) to achieve $O(\log n)$-depth circuits
for $O(n)$ inputs. Many functions, most importantly binary
addition and sharpening estimated phase values to exponential precision,
can ingeniously be recast in this parallelized FA setting.
First, $n$ copies are created of the FA
transition function table. Each copy is then narrowed by its corresponding
input,
composing adjacent function tables in a binary tree hierarchy, and applying
these composed functions to ``skip ahead'' in the FA and compute
intermediate states in parallel.

However, while achieving good asymptotic bounds, this approach has some
disadvantages when applied to quantum adding: an overly-general narrowing
procedure, and the inability to add in place without resorting to
Bennett's uncomputing trick \cite{bennett}.
For these reasons, we use the quantum carry-lookahead
adder (QCLA) presented in \cite{dkrs}. It follows the spirit of parallel iteration
using classical carry-lookahead techniques, in essence ``hardcoding''
the narrowing and composing steps above for the carry bits generated
and propagated in binary addition.

The resource counts for QCLA on two $n$-bit numbers are shown below,
for both in-place and out-of-place addition. The output bits for
out-of-place addition are counted as ancillae for consistency.

\begin{tabular}{|r|c|c|}
\hline
Resource & QCLA-IN & QCLA-OUT\\
Depth    &

$\lfloor \log n \rfloor  + \lfloor \log (n-1) \rfloor +
\lfloor \log n/3 \rfloor + \lfloor \log (n-1)/3 \rfloor + 14$ &
$\lfloor \log n \rfloor + \lfloor \log n/3 \rfloor + 7$ \\

Toffoli Gates &
$10n - 3w(n) - 3w(n-1) - 3\lfloor \log n \rfloor -
3 \lfloor \log(n-1) \rfloor - 7$ &
$5n - 3w(n) - 3 \lfloor \log n \rfloor - 1$\\

CNOT Gates & $4n-5$ & $3n-1$\\

NOT Gates & $2n-2$ & $0$ \\

Ancillae & $2n - w(n) - \lfloor \log n \rfloor - 1$ & $n$\\
\hline
\end{tabular}

By combining Super-Kitaev's encoded adding with the QCLA, we can now
add $m$ numbers of $n$-bits each with a circuit of size $O(mn)$ and
depth $O(\log m + \log n)$.

We denote the final resource counts of adding as the following:

\begin{eqnarray*}
\text{ADD-IN}(m, n) & = & \text{QCLA-IN}(n) + \text{ENC-ADD-IN}(m,n)\\
\text{ADD-OUT}(m, n) & = & \text{QCLA-OUT}(n) + \text{ENC-ADD-OUT}(m,n)\\
\end{eqnarray*}

\subsection{Quantum Multiplication}
\label{subsec:mult}

Now that we know how to add quantum numbers efficiently, we can use this
to implement a similarly low-depth multiplication circuit.

Consider the inputs to our quantum multiplier as two $n$-bit numbers,
$a$ and $b$.
Our grade school multiplication algorithm simply sums up $n$ copies of $a$,
where $c(i)$ denotes $a$ shifted up by $i$ bits and multiplied by bit $b_i$.

\begin{figure}
\caption{Piecewise definition of $c(i)$ shifted multiples for multiplication}
\end{figure}

There are $n$ such shifted multiples $c(i)$. By summing them up and retaining
only the low-order $n$-bits, we are reducing multiplication of two $n$-bit
numbers, modulo $2^n$, to the addition of $n$ $n$-bit numbers, which we
already know how to do in parallel. In fact, we can do it in
$\log(n) + \log(n) = O(\log n)$ depth.
To hold the shifted multiples, we will
need $n(n+1)/2$ ancillae, since each bit shift leaves a low-order zero bit
that we don't need to include in our sum.

We define the resource counts of in-place multiplication as follows.

\begin{displaymath}
MULT-2(n) = ADD(n, n) + n(n+1)/2 \text{ancillae}
\end{displaymath}

\section{Quantum Divider}

The most resource-intensive step of the registered phase shift operation
is solving $p = p(s,l)$ for a given resulting
eigenvector $\ket{\psi_{2s-1}}$ after phase estimation and a desired
phase shift $exp(2\pi i l / 2^n)$. Solving the equation
$(2s-1)p \equiv \mod 2^n$ involves finding a modular inverse, which
we can convert to the following product.

To solve for $p$, we have

\begin{displaymath}
p = (2s - 1)^{-1} l \mod 2^n
\end{displaymath}

To find $y = (2s-1)^{-1} \mod 2^n$, it should have the property
$y \cdot (2s-1) \equiv 1 \mod 2^n$ or equivalently
$-y \cdot (2s-1) \equiv -1 \mod 2^n$. As a guess, we will choose the
$-y$ version and have it be a fraction with $(2s-1)$ in the denominator and
something in the numerator which is $1$ less than a multiple of
$2^n$, such as $2^n s^n - 1$.
We choose this to resemble the closed form of a geometric series,
as shown below.

\begin{displaymath}
-y = \frac{2^n s^n - 1}{2s - 1} = \frac{(2s)^n - 1}{(2s) - 1} =
\frac{1 - (2s)^n}{1 - (2s)} = \sum_{i=0}^n (2s)^i
\end{displaymath}

We use the following observation to convert this geometric series into a
product of sums involving power-of-two exponents, which makes for 
easy multiplication via bitshifting.

\begin{displaymath}
\sum_{k=0}^{2^l - 1} z^k = (1 + z)(1 + z^2)(1 + z^4)\cdots(1 + z^{2^{l-1}})
\end{displaymath}

In general a geometric series with $m = 2^l$ terms
involves $m$ sums
and $m$ products. Using the above formula, we have $\log m$ products and
$\log m$ sums. Both are necessary to achieve our desired logarithmic
depth for the overall registered phase shifting procedure.

Substituting, we get:

\begin{displaymath}
-y = \prod_{r=0}^{\lceil \log_n \rceil - 1} (1 + (2s)^{2^r})
\end{displaymath}

Then the final solution of $p$ in the first equation above is:

\begin{displaymath}
p = -yl \mod 2^n = -l \prod_{r=1}^{\lceil \log n \rceil -1} (1 + (2s)^{2^r})
\end{displaymath}

The resource counts for this modular division is equivalent to:

\begin{displaymath}
\text{DIV-MOD}(n) = \lceil \log n \rceil \cdot \text{MULT-2}(n) +
 (\lceil \log n \rceil - 1) \cdot \text{ADD-IN}(2, n)
\end{displaymath}

