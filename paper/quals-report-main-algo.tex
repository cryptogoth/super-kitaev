\subsection{The Main Algorithm}
\label{subsec:main}

Now assuming we have parallelized phase estimation as a black box, here are
the steps to the Super-Kitaev algorithm.

\begin{theorem}[Theorem 13.5]
Any circuit $C$ of size $L$ and depth $d$ over a fixed finite basis can be
simulated with precision $\delta$ by an $O(Ln + n^2\log n)$-size
$O(d \log n + (\log n)^2)$-depth circuit $C'$ over the standard basis, where
$n = O(\log(L/\delta))$.
\end{theorem}

\begin{proof}

For the steps below, we note that we can avoid solving the equation $kp \equiv l (mod 2^n)$
if we set $k=1$. This is equivalent to realizing $\Upsilon_n(e^{2\pi i / 2^n})$
by applying $\Upsilon(A)$ from Section \ref{subsec:phase-shift} to the state
$\ket{\psi_{n,1}}$. So we need to create one copy of this magic state to
simulate each $\Lambda(e^{i\phi})$ gate, possibly up to $L'$ of them.

\begin{enumerate}
\item Precompile the circuit into gates from $\mathcal{Q} \cup \{\Lambda(e^{i\phi})\}$
using the results from Section \ref{subsec:precompile}.
\item Create the state magic state $\ket{\psi_{n,0}} = H^{\otimes n}\ket{0^n}$
\item Turn it into $\ket{\psi_{n,1}} = \Upsilon(e^{-2\pi i / 2^n}) \ket{\psi_{n,0}}$
using the procedure in Section \ref{subsec:phase-shift}
This is done with a circuit of size $O(n^2\log n)$ and $O((\log n)^2)$ depth.
\item Make $L'$ copies of the state $\ket{\psi_{n,1}}$ out of one copy by 
apply the addition operation below.
\item Simulate the circuit $C'$ using one copy of $\ket{\psi_{n,1}}$ per gate,
this takes size $O(n)$ and depth $O(\log n)$.
\item Reverse the first three steps.
\end{enumerate}

To copy the state $\ket{\psi_{n,k}}$ it suffices to apply the following
operator:

\begin{equation}
\ket{\psi_{n,k}}^{\otimes m} = W^{-1}\left( \ket{\psi_{n,0}}^\otimes(m-1) \otimes \ket{\psi_{n,k}} \right)
\end{equation}

where $W$ is defined by

\begin{equation}
W : \ket{x_1,\ldots,x_{m-1},x_m} \rightarrow \ket{x_1,\ldots,x_{m-1},x_1+\ldots+x_m}
\end{equation}

\end{proof}
