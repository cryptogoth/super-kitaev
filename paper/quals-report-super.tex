\section{The Super-Kitaev Algorithm}

Our development of the Super-Kitaev algorithm follows closely the one in the
book by Kitaev, Shen, and Vyalvi \cite{ksv02}, although of course, it is
simply called Theorem 13.5. Likewise, we will cite the theorem / lemma numbers
from the book, which contains a self-contained description of all the parts
needed for Super-Kitaev.

The basic steps of this method are the following:

\begin{enumerate}
\item Precompiling to get $L'$ gates in $\mathcal{Q} \cup \Lambda(e^{i\phi})$.
Here we see that there is an
efficient decomposition of several important gates into the standard set and
controlled-phase-shifts.
\item Implementing a controlled-phase-shift. This ends up being the hardest
part of simulating a quantum circuit, and requires each of the remaining steps.
\item Use phase estimation of an addition operator to magic states in order to
enact the desired phase shift. This requires both efficient creation of
{\em magic states} and an efficient {\em quantum adder} circuit.
\item Create one magic state, and then make $L'$ copies of it. Use one copy
to simulate each $\Lambda(e^{i\phi})$ gate.
\end{enumerate}
