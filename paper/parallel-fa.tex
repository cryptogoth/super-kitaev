\documentclass{article}

%\usepackage{times}
%\usepackage{fullpage}
%\usepackage{eprint}
%\usepackage{rotating}
%\usepackage{eepic}
\usepackage{amsfonts}
%\usepackage{algorithmic}
\usepackage{amsthm}
%\usepackage{xy}

%\tocdetails{%
%   volume=0, number=5, year=2005, firstpage=78,
%   received={October 15, 2004}, 
%   published={December 31, 2004}
%}

\theoremstyle{plain}
\newtheorem{theorem}{Theorem}
%\newtheorem{idea}{Idea}[section]

%\theoremstyle{definition}
%\newtheorem{thought}[theorem]{Thought}
%\newtheorem{thought}{Thought}[section]

\title{Parallel Iteration of Finite Automata}
\date{January 25, 2011}
\author{Paul Pham}

\input{Qcircuit}

\begin{document}

%\newcommand{\ket}[1]{|#1 \rangle}
%\newcommand{\bra}[1]{\langle #1 |}
\newcommand{\braket}[2]{\langle #1|#2 \rangle}
\newcommand{\normtwo}{\frac{1}{\sqrt{2}}}
\newcommand{\norm}[1]{\parallel #1 \parallel}

\maketitle

\section{Abstract}

We describe the technique of parallel iteration of classical finite
automata (FA) to achieve exponential improvement in circuit depth with
only polynomial circuit size overhead. Running an FA on $n$ inputs
normally takes $O(n)$ depth, but by copying the transition function
table, narrowing it by the inputs (known \emph{a priori}), composing
adjacent functions, and then applying composed functions to iterate
the FA on several states in parallel, we can achieve $O(\log n)$-depth.
In particular, we describe how this technique is used for quantum
adding and classical phase sharpening in the Super-Kitaev algorithm
for quantum compiling, giving exact gate sequences and the resource
requirements in each case.


\section{Introduction}
\label{sec:intro}

In \cite{ksv02}, the authors describe the development of parallelized
iteration of finite automata, which they later use in a low-depth, highly-
parallel simulation of quantum gates (quantum compiling). We call this
quantum compiling
scheme the Super-Kitaev procedure in contrast to the normal
Solovay-Kitaev construction as described in \cite{nc00}.
However, in these notes, we describe
this method of parallelized iteration as a useful tool in its own right,
even if you don't care about quantum compiling or quantum computing at all.
Finite automata (FA) are useful for computing all kinds of classical functions,
which is how they are used in Super-Kitaev, albeit in a reversible fashion
with the possibility of feeding in quantum inputs and getting out quantum
outputs, in the most general case.
But the idea of parallelizing FA iteration can also be used in an
efficient classical implementation that could be parallelized on modern
multi-core processors. It also demonstrates that we can achieve an
exponential speedup in depth (from linear depth of normal, serial iteration to
logarithmic depth of parallel iteration) using only a polynomial overhead in
gates required in the circuit model.

These notes will develop our general (quantum) model of finite automata, the actual
sequence of gates required, and resource requirements for circuit size,
circuit depth, and ancillae qubits required. We will then
apply this model to two particular instances of FA-computable functions useful in
Super-Kitaev: adding and phase sharpening. We will calculate the actual resources
required for a wide range of input sizes to give us some intuition about the
asymptotic tradeoffs and multiplicate constants involved. These are normally
neglected in theoretical descriptions, but are of interest to engineers
when it comes time to design and build an actual implementation. We will closely
follow the development in \cite{ksv02}, to which the interested reader is
referred for a more concise, self-contained description of the Super-Kitaev
algorithm and its sub-modules.


\section{Reversible Finite Automata and Quantum Inputs}

We don't need to do much to a classical finite automata so that it will
accept quantum input symbols other than making the transition function
reversible. Everywhere below we will assume we have this reversibleness and
use reversible gates like CNOT and Toffoli with the understanding that these
would also work on classical inputs. Also, note that the transition functions
for all the FA we care to compute so far are purely classical, in that the
they can be represented as a table with purely zeros and ones. The only
potential quantumness comes from the inputs to the FA.

To compare resources
between the classical and quantum cases, we will use the (reversible) circuit
model and treat circuit depth as analogous to the running time on a fully
parallelized machine.

\subsection{Reversible Finite Automata for Decision Problems}

The traditional definition of an FA is the following tuple.

\begin{displaymath}
(\Sigma,S,s_0,f,S')
\end{displaymath}

\begin{description}
\item[$\Sigma$] is the alphabet of input symbols
\item[$S$] is the set of states
\item[$s_0$] is the start state
\item[$f$] $: S \times \Sigma \rightarrow S$ is the
transition function that determines the next state given the current state
and current input symbol
\item[$S'$] is the subset of final accept states.
\end{description}

We run the machine for $n$ steps (\emph{iterations}),
starting in $s_0$ and feeding it one
input symbol $a_i$ at a time from the set $\{a_1,\ldots,a_n\}$.
If the resulting state $s_n$ is
in $S'$, we accept, otherwise we reject.

Normally, this takes us $O(n)$ gates executed in sequence, so $O(n)$ depth.
This is because for step $i$, we need to know the input state $s_{i-1}$ in
order to compute the next state $s_i$.

However, since we know all $n$ input symbols $\{a_i\}$ at the beginning, it is
possible to iterate in parallel to compute $s_i$'s simultaneously in a
logarithmic-depth, binary-tree fashion. This is a simple idea which I'll spend
the rest of these notes developing. We are trading some overhead to add
this parallelism, but only a polynomial amount compared to the exponential
speedup of $O(\log n)$ depth.

\subsection{Reversible Finite Automata for Computing Binary Functions}

I will modify the FA tuple above to be the following.

\begin{displaymath}
(\sigma,S,s_0,f,g)
\end{displaymath}

\begin{description}
\item[$\sigma$] $= \{0,1\}^{k_i}$, the input alphabet restricted to symbols
encoded in $k_i$ bits.
\item[$S$] $= \{0,1\}^{k_s}$, the set of states restricted to the encodings of
$k_s$ bits.
\item[$s_0$] the initial state as before, $s_0 \in \{0,1\}^{k_s}$
\item[$f$] $:S \times \Sigma \rightarrow S$ the transition function as before
\item[$g$] $:S \times \Sigma \rightarrow \{0,1\}^{k_o}$ an optional output
function that outputs a symbol encoded in $k_o$ bits.
\end{description}

Namely, I've restricted all symbols and states to be encodings in bits
and added
an optional output function $g$ so that at each step, we can produce an output symbol.
Often however, the destination state of each iteration $s_i$
will also be treated as our output for computing some function, in addition to
being an input to the next iteration.
We will drop the accept states, since in general we are interested in
computing binary functions instead of just
rejecting/accepting. However, none of these changes affects the computational
power of finite automata.

The transition function $f$ takes the form of a table  of
$2^{k_i+k_s}$ lines, one for each combination of state and input symbol.

Our goal is to compute $f$ on all $n$ inputs to produce all intermediate
states $\{s_i\}$ as a ``trace'' or history of the FA execution,
and optionally $g$ at each iteration to produce $n$ outputs.

Now this FA can compute any pairs of functions $f$ and $g$ defined as:

\begin{eqnarray}
f & : &\{0,1\}^{k_i+k_s} \rightarrow \{0,1\}^{k_s} \\
g & : & \{0,1\}^{k_i+k_s} \rightarrow \{0,1\}^{k_o}
\end{eqnarray}

The reason for this rather artificial separation of the function-to-be-computed
into $f$ and $g$ is that some FA (namely Kitaev's parallelized adder) compute
an output which is not used to determine the next state at all, and for
bookkeeping purposes it is more convenient to treat such an output separately.

\subsection{Classical Control}

In the most general setting, we have a classical controller, a separate
computer which generates gates to apply to our target finite automata,
which may be running on a quantum computer on quantum inputs.

If we know our classical inputs in advance, it may be possible to
improve the resource counts below by precomputing gates on our classical
controller, optimized to our particular FA. However, for general quantum
inputs, we must generate a fixed, universal gate set to run our
reversible FA. This is what we develop first in Section \ref{sec:pi}, which
is generally overly expensive, and then we improve the cost in
Sections \ref{sec:add} and \ref{sec:sharp} based on our knowledge
of the specialized structure of adding and phase sharpening, respectively.

This separation of a classical controller and a target machine is also
useful because this is how we imagine most realistic quantum machines
will run, and it will always be cheaper to precompute as many
classical operations as possible on the classical controller. For
example, in Subsection \ref{subsec:narrow} we encode our FA function
tables exclusively when our classical controller determines which
gates to use in narrowing the function tables by input bits.


\section{Notation}
\label{sec:notation}

Because the $g$ output function is optional, we imitate the Kronecker delta
function to simplify the resource counts with a single multiplicative
constant:

\begin{equation*}
\delta_g = \left\{
  \begin{array}{rl}
    0 & \text{if } g \text{ is unused}\\
    1 & \text{if } g \text{ is used}\\
  \end{array} \right.
\end{equation*}

We denote by $w(f)$ for a particular function table $f$ to be the
number of ones that appear in the outputs. If the optional output function $g$
is not needed by an FA, then we have $w(g) = k_o = 0$.
The number of ones can be at most the number of lines in the function
table, so $w(f) \le 2^{k_i+k_s}$.

Similarly, we denote by $z(f)$ for a particular function table $f$ to be
the number of ones which appear in the outputs for a zero input. Again,
an unused function $g$ would result in $z(g) = 0$.
Naturally, we can't have more output ones for zero inputs than
output ones for any inputs: $z(f) \le w(f)$.

Since $n$ may not be a power of two, a common parameter will be the
logarithm of $n$ rounded up to the nearest power of two, which we'll
defined as $l = \lceil \log_2 n \rceil$.

In the gate counts below, we will include the counts of single-qubit NOT, or X,
gates, which is used to negate the control of controlled-NOTs or Toffoli's.

\begin{displaymath}
\Qcircuit @C=2em @R=1.5em {
& \ctrlo{1} & \qw & =  & & \qw & \targ & \ctrl{1} & \qw &\\
& \targ     & \qw &    & & \qw & \qw   & \targ    & \qw &
}
\end{displaymath}

\begin{displaymath}
\Qcircuit @C=2em @R=1.5em {
& \ctrlo{1} & \qw & =  & & \qw & \targ & \ctrl{1} & \qw &\\
& \ctrl{1}     & \qw &    & & \qw & \qw   & \ctrl{1}    & \qw &\\
& \targ     & \qw &    & & \qw & \qw   & \targ    & \qw &
}
\end{displaymath}

To compare the relative complexity of implementing the Toffoli gate versus
CNOT and single-qubit gates, we can consider the following implementation
of Toffoli given in Figure 4.9, page 182 of \cite{nc00}. One Toffoli equals
6 CNOT gates and 10 single-qubit gates (Hadamard and controlled-phase).

We denote $d_T$ as the depth of a Toffoli gate, $d_C$ as the depth of a
CNOT, and $d$ as the depth of a single-qubit gate. That last one we admit
can vary widely based on the particular single-qubit gate, but we have to
draw the line at some point. So based on our implementation of the
Toffoli above, we have:

\begin{displaymath}
d_T = 6d_C + 10d
\end{displaymath}

We also note that a multiply ($n$) controlled-NOT ($\Lambda^n(X)$) can be
implemented with $n-1$ Toffoli's and $n-2$ ancillae, where $n=2$ is a
normal Toffoli.

In each of the four steps of parallel iteration described in Section
\ref{sec:pi}, we provide a table of resource usage: the number of gates
of each type (single-qubit NOT, CNOT, and Toffoli), the number of ancillae,
and the depth of the circuit. The number of gates is specified per input and
often redefined as $\ell$ with some subscript, which we then reuse in
the circuit depth for notational convenience. This is because
these operations happen in parallel for each of the inputs, leading to an
implicit factor $n$. However, this factor disappears or becomes logarithmic
in the circuit depth; otherwise, we would not have accomplished our goal
of decreasing depth with parallel iteration!
There is also always a factor of 2 in the depth
and gate count to allow for uncomputing the ancillae back to $\ket{0}$.


\section{Parallel Iterations}
\label{sec:pi}

The process of parallelizing iterations has four main steps, each of which
has its own section below with general resource requirements.

The first step is copying function tables due to fanout restrictions.
For classical circuits this is a polite gesture towards the fact that
transistors can only drive a finite (but fairly large) load. For quantum
circuits, this is an absolute necessity. So if we are going to be applying
a bunch of FA transition functions in parallel, we'd better have a separate
copy of the function for each execution.

The second step is narrowing the function tables by input. Since we know all
the input symbols in advance, we can vastly simplify the table for each
iteration $i$ by only selecting the lines which correspond to the input
symbol $a_i$. In a sense, we are ``optimizing'' the table, or hard-coding
for a particular input symbol by ``baking'' it into the execution of this new,
narrower function.

The third step is composing adjacent pairs of these narrowed function tables
in a binary tree hierarchy, so that we can skip ahead to future iterations.

Finally, the last step is to apply these composed function tables, in the
same binary tree hierarchy, to compute all the states $s_i$ in parallel
using only logarithmic depth.

\subsection{Copying Input Bits}
\label{subsec:copy}

This is the simplest step and results in $n$ copies of the input bits. Since
these are the control bits for many subsequent operations, if we wish to
source them in parallel, we must have multiple copies. This simply
involves a CNOT to copy each input into an empty ancilla.

We copy in a parallel fashion to achieve logarithmic depth, and each new
copy can itself produce another copy simultaneously. Therefore the circuit depth
is just the height of
a binary tree with $n$ nodes, times two for uncopying later.

\begin{table}
\begin{center}
\begin{tabular}{|r|cc|}
\hline
CNOT Gates Per Input &  $\ell_C = $ & $k_i \cdot 2$ \\
Ancillae & & $n\cdot k_i$\\
Circuit Depth & & $(\lceil \log_2 n \rceil - 1 )\cdot \ell_C \cdot d_C$\\
\hline
\end{tabular}
\caption{Resource counts for Step 1: copying input bits in parallel.}
\end{center}
\end{table}


\subsection{Narrowing Function Tables by Inputs}
\label{subsec:narrow}

Now that we have copied enough inputs to source, we can use them
to narrow our function tables (both $f$ and $g$) by successively
halving the function tables based on the value of each
input bit. This is where we rely on our classical control to encode
the function tables in the Toffoli's applied in the narrowings. The function
table itself appears nowhere else in our circuits.

Therefore, each of the $k_i$ input bits takes two Toffoli's,
one of which has a NOT gate to negate its control bit. This
successive halving takes the following number of ancillae, since we start
with a function table of $2^{k_i + k_s}$ and want to end with a function
table of $2^{k_s}$:

\begin{displaymath}
2^{k_i + k_s - 1} + 2^{k_i + k_s - 2} + \ldots + 2^{k_s + 1} + 2^{k_s}
\end{displaymath}

which we can express as the difference of sums:

\begin{displaymath}
\sum_{j=0}^{k_i + k_s - 1} 2^j - \sum_{j=0}^{k_s} 2^j = 
\left( 2^{k_i+k_s} - 1 \right) - \left( 2^{k_s+1} - 1 \right)
\end{displaymath}

Likewise if we use the output function $g$, we can treat each of its
$k_o$, output bits as a separate one-bit output function, and then we
require the same number of gates and ancillae as above for each one.

There is one such narrowed table copy of $f$ (and $g$) for each of $n$ inputs.
For each copy of the table, for each of $k_i$ successive narrowings and
input bits, there are several gates that need to be applied serially because
they use the same control bits---this makes up the circuit depth.
These serial gates are one Toffoli for each of $w(\cdot)$ one outputs,
$z(\cdot)$ of which require single-qubit NOT gates to negate their control.
There is also a factor
of 2 for uncomputing all of these ancillae back to $\ket{0}$.

\begin{table}
\begin{center}
\begin{tabular}{|r|cc|}
\hline
NOT Gates per Input & $\ell_N = $ & $[z(f)+z(g)] \cdot (2^{k_i+k_s} - 2^{k_s+1}) \cdot 2$ \\
Toffoli Gates per Input & $\ell_T = $ & $[w(f)+w(g)] \cdot (2^{k_i+k_s} - 2^{k_s+1}) \cdot 2$\\
Ancillae & & $n\cdot (2^{k_i+k_s} - 2^{k_s+1}) \cdot (1 + \delta_g2\cdot k_o) \cdot 2$\\
Circuit Depth & & $\ell_T d_T + \ell_N d$\\
\hline
\end{tabular}
\caption{Resource counts for Step 2: narrowing function tables by inputs in parallel}
\end{center}
\end{table}

\subsection{Composing Narrowed Function Tables}
\label{subsec:compose}

Now that we have narrowed the function tables by input, they only depend
on the current state. By composing adjacent function tables in a binary
tree hierarchy, we can compute all the intermediate states of an FA
execution while still preserving logarithmic depth.

We will use the notation $F_{k,i}$ to denote the $i$th composed function table
at the $k$th level of recursion, according to the diagram below for $n=4$.
$F_{0,i}$, at the base level, corresponds to the original function $f$
narrowed by input $a_i$.

\begin{displaymath}
\xymatrix @C=1em {
  \ar[rr]^{s_0} & &
   *+[F]{F_{0,0}} \ar[rr]^{s_1} & &
   *+[F.]{F_{0,1}} \ar[rr]^{s_2} & &
   *+[F]{F_{0,2}} \ar[rr]^{s_3} & &
   *+[F.]{F_{0,3}} \ar[rr]^{s_4} & & \\
  & & & *+[F]{F_{1,0}} \ar@{<-}[ul] \ar@{<-}[ur]
  & & & & *+[F.]{F_{1,1}} \ar@{<-}[ul] \ar@{<-}[ur] \\
  & & & & & *+[F]{F_{2,0}} \ar@{<-}[ull] \ar@{<-}[urr]
}
\end{displaymath}

How many composition operations do we need to do for our $n$ initial
narrowed $F_{0,i}$ tables? The resulting full binary tree (except for
the first level) has $\lceil \log_2 n \rceil = l$ levels and therefore
$2^{l}-1$ nodes. However, as will become apparent in the next section, we don't
need the compositions with a dotted border above.

Since at each level of compositions, we can remove half of the nodes
(except for the root node), we have the following formula of the
total number of compositions we need for $n$ iterations and $l$ levels,
where the root level corresponds to $j=0$.

\begin{equation}
\sum_{j=0}^{l-1} 2^j - \sum_{j=1}^{l-1} 2^{j} =
(2^l - 1) - (2^{l-1} - 1) = 2^l - 2^{l-1} = 2^{l-1}
\end{equation}
\label{eqn:compose}

\subsubsection{Composing Functions with Single-Bit Output}

First we will show how to compose a function table with a single-bit output,
since this is the simplest case. However, $f$ and $g$ may be multi-bit output
functions ($k_s, k_o > 1$). To solve this, we just decompose $f$ and $g$
into a bunch of single-bit functions, and then show how to combine these
functions so that we end up composing the original, multi-bit functions.

For the single-bit case, let's concentrate on $f$ which has $k_s = 1$
and $2^{k_s} = 2$ lines.
(the treatment of $g$ is equivalent with $k_o = 1$ and also $2^{k_o} = 2$ lines).

This is the circuit for composing two such functions, $F_{k,i}$ and $F_{k,i+1}$,
into a new function $F_{k+1,\lfloor 1 \rfloor}$
which is equivalent to first applying
$F_{k,i}$ to get an output and feeding that into $F_{k,i+1}$. In circuit below,
the ``input'' is actually the preceding state $s_i$.

\begin{displaymath}
\Qcircuit @C=2em @R=1.5em {
F_{k,i}^j  & & s_i = 0 & & \qw & \ctrl{3} & \qw      & \ctrlo{2} & \qw       & \qw \\
           & & s_i = 1 & & \qw & \qw      & \ctrl{2} & \qw       & \ctrlo{1} & \qw \\
F_{k,i+1}^j & & s_i = 0 & & \qw & \qw      & \qw      & \ctrl{2}  & \ctrl{3}  & \qw \\
           & & s_i = 1 & & \qw & \ctrl{1} & \ctrl{2} & \qw       & \qw       & \qw \\
F_{k+1,i/2} & & s_i = 0 & & \qw & \targ    & \qw      & \targ     & \qw       & \qw \\
 & & s_i = 1 & & \qw & \qw      & \targ    & \qw       & \targ     & \qw \\
}
\end{displaymath}

\subsubsection{Composing Functions with Multiple-Bit Output}

Now assuming that $k_s > 1$ and we have decomposed such a function table into
$k_s$ single-bit output functions, how do we combine them together again?
We consider the case for $k_s = 2$, shown below, and it is easy to generalize
this for higher $k_s$.
Each $s_i$ is a 2-bit state
where $s_{i,0}$ is the first bit and $s_{i,1}$ is the second bit.
Each $F_i^j$ is a single-bit output function that supplies bit $j$
from iteration $i$ to feed into the next iteration $i+1$.

\xymatrix{
          & *+[F]{F_{i,0}} \ar[r] & s_{i+1,0} \ar[r] \ar[ddr] & *+[F]{F_{i+1,0}} \ar[r] & s_{i+2,0} \\
s_i \ar[ur] \ar[dr] &              &                     &           \\
          & *+[F]{F_{i,1}} \ar[r] & s_{i+1,1} \ar[r] \ar[uur] & *+[F]{F_{i+1,1}} \ar[r] & s_{i+2,1} \\
}

The circuit for composing three of these functions is given below, for
the input state $s_i$ equal to $00$ and $01$. Similar gates can be found
for the remaining state values of $10$ and $11$.
$F_i^0$ and $F_i^1$ supply two bits which become the input for
$F_{i+1}^0$. When composed into the new function $F_{i+1}^0$,
we have a table which produces the first bit of the state $s_{i+2}$
give all the bits of state $s_{i}$.

\begin{displaymath}
\Qcircuit @C=2em @R=1.5em {
F_{k,i}^0     & & s_i = 00 & & \qw & \ctrl{4} & \ctrlo{4}   & \ctrl{4} & \ctrlo{4}  & \qw      & \qw         & \qw       & \qw       & \qw \\
              & & s_i = 01 & & \qw & \qw      & \qw         & \qw       & \qw       & \ctrl{4} & \ctrlo{4}   & \ctrl{4} & \ctrlo{4}  & \qw \\
              & & s_i = 10 & & \qw & \qw      & \qw         & \qw       & \qw       & \qw      & \qw         & \qw       & \qw       & \qw \\
              & & s_i = 11 & & \qw & \qw      & \qw         & \qw       & \qw       & \qw      & \qw         & \qw       & \qw       & \qw \\
F_{k,i}^1     & & s_i = 00 & & \qw & \ctrl{7} & \ctrl{5}    & \ctrlo{6} & \ctrlo{4} & \qw      & \qw         & \qw       & \qw       & \qw \\
              & & s_i = 01 & & \qw & \qw      & \qw         & \qw       & \qw       & \ctrl{6} & \ctrl{4}    & \ctrlo{5} & \ctrlo{3} & \qw \\
              & & s_i = 10 & & \qw & \qw      & \qw         & \qw       & \qw       & \qw      & \qw         & \qw       & \qw       & \qw \\
              & & s_i = 11 & & \qw & \qw      & \qw         & \qw       & \qw       & \qw      & \qw         & \qw       & \qw       & \qw \\
F_{k,i+1}^0   & & s_i = 00 & & \qw & \qw      & \qw         & \qw       & \ctrl{4}  & \qw      & \qw         & \qw       & \ctrl{5}  & \qw \\
              & & s_i = 01 & & \qw & \qw      & \ctrl{3}    & \qw       & \qw       & \qw      & \ctrl{4}    & \qw       & \qw       & \qw \\
              & & s_i = 10 & & \qw & \qw      & \qw         & \ctrl{2}  & \qw       & \qw      & \qw         & \ctrl{3}  & \qw       & \qw \\
              & & s_i = 11 & & \qw & \ctrl{1} & \qw         & \qw       & \qw       & \ctrl{2} & \qw         & \qw       & \qw       & \qw \\
F_{k+1,i/2}^0 & & s_i = 00 & & \qw & \targ    & \targ       & \targ     & \targ     & \qw      & \qw         & \qw       & \qw       & \qw \\
              & & s_i = 01 & & \qw & \qw      & \qw         & \qw       & \qw       & \targ    & \targ       & \targ     & \targ     & \qw \\
              & & s_i = 10 & & \qw & \qw      & \qw         & \qw       & \qw       & \qw      & \qw         & \qw       & \qw       & \qw \\
              & & s_i = 11 & & \qw & \qw      & \qw         & \qw       & \qw       & \qw      & \qw         & \qw       & \qw       & \qw \\
}
\end{displaymath}

To compute each bit of the next state
$s_{i+1}$, a multiple-bit composition consists of 
$k_s$ single-bit compositions. Each single-bit composition is
composed of $2^{k_s}$ generalized Toffoli's with $k_s +1$ controls, and
$2^{k_s}$ NOT gates for negative controls. Remember that a generalized
Toffoli with $k_s + 1$ controls requires $k_s$ Toffoli's and $k_s - 1$
ancillae. Each single-bit composition requires $2^{k_s}$ ancillae to
hold the new function table. The single-bit compositions cannot be done
in parallel, however, because they all use the same control bits.

From the beginning of this section, we have that there are $2^l - 1$
multiple-bit compositions, in general, which multiplies everything above,
except the circuit depth. The circuit depth multiplier is
just the height of the tree of compositions is $l$.

\begin{table}[!h]
\begin{center}
\begin{tabular}{|r|cc|}
\hline
NOT Gates per Multiple-Bit Composition & $\ell_N = $ & $k_s \cdot 2^{k_s} \cdot 2$ \\
Toffoli Gates per Multiple-Bit Composition & $\ell_T = $ & $k_s \cdot (2^{k_s}\cdot (k_s-1)) \cdot 2$\\
Total NOT Gates & & $(2^{l-1}) \cdot \ell_N$ \\
Total Toffoli Gates & & $(2^{l-1}) \cdot \ell_T$ \\
Ancillae & & $(2^l - 1)\cdot k_s \cdot (2^{k_s} + (k_s - 1)) \cdot 2$\\
Circuit Depth & & $l\cdot(\ell_T d_T + \ell_N d)$\\
\hline
\end{tabular}
\caption{Resource counts for Step 3: composing narrowed function tables in parallel.}
\end{center}
\end{table}

\subsection{Applying Function Tables}
\label{subsec:apply}

Finally, we have narrowed tables composed so that we can skip ahead to
intermediate states in a logarithmic-depth, binary tree fashion.

The scheme for determining all the intermediate states $s_i$ follows
the diagram below. Recall that we start out knowing the initial state
$s_0$.

\begin{displaymath}
\xymatrix @C=1em {
\text{level 1} & s_0 \ar[rrrr]^{F_{2,0}} \ar[ddd] &     &        &     & s_4 \ar[ddd]\\
\text{level 2} &     \ar[rr]^{F_{1,0}}   &     & s_2 \ar[dd]    &     & \\
\text{level 3} &     \ar[r]^{F_{0,1}} & s_1 \ar[d] & \ar[r]^{F_{0,3}} & s_3 \ar[d] & \\
              &               &     &        &     &
}
\end{displaymath}

In general, each composed function has multiple-bit output, and to
apply it we actually need to apply $k_s$ separate single-bit output
functions. Applying each single-bit output function takes $2^{k_s}$
generalized Toffoli's,
each with $k_s+1$ controls (one for every bit of the input, plus one for
the function table to apply) and in total requiring $2^{k_s}$ negative
controls. Recall that such a generalized Toffoli can be implemented
with $k_s - 1$ normal Toffoli's and $k_s - 2$ ancillae.
Finally, to hold the final value of the $s_i$'s, we need $k_s$ ancillae.

\begin{displaymath}
\Qcircuit @C=2em @R=1.5em {
F_{k,i}^j & & s_i = 00 & & \qw & \ctrl{4} & \qw      & \qw      & \qw      & \qw\\
        & & s_i = 01 & & \qw & \qw       & \ctrl{3}  & \qw       & \qw      & \qw\\
        & & s_i = 10 & & \qw & \qw       & \qw       & \ctrl{2}  & \qw      & \qw\\
        & & s_i = 11 & & \qw & \qw       & \qw       & \qw       & \ctrl{1} & \qw\\
s_i^0   & &          & & \qw & \ctrlo{1} & \ctrlo{1} & \ctrl{1}  & \ctrl{1} & \qw \\
s_i^1   & &          & & \qw & \ctrlo{1} & \ctrl{1}  & \ctrlo{1} & \ctrl{1} & \qw \\
s_{i+1}^j & &        & & \qw & \targ     & \targ     & \targ     & \targ    & \qw \\
}
\end{displaymath}

Unlike in the previous step with compositions, we need to apply the
lowest level of our function tables, the $F_{0,i}$ functions. Therefore,
in analogy to Equation \ref{eqn:compose},
the total number of function applications is given below. The
circuit depth depends on $l$ levels of applications, and not $l+1$
because the two highest levels (to compute $s_n$ and $s_{n/2}$) can be
done in parallel.

\begin{equation}
\left( \sum_{j=0}^{l} 2^j \right) - \left( \sum_{j=1}^{l} 2^j \right)
= (2^{l+1} - 1) - (2^{l} - 1) = 2^l = n
\end{equation}

Also unlike with compositions, we bring back our optional output function
$g$, which we have already narrowed and is ready to be applied at each
of $n$
iteration with the states $s_i$ which we have just computed in this step.
Applying $g$ is the same as applying $f$ (and $F_{k,i}$) above.

\begin{table}
\begin{center}
\begin{tabular}{|r|cc|}
\hline
NOT Gates per Application & $\ell_N = $ & $2^{k_s} \cdot 2$ \\
Toffoli Gates per Application & $\ell_T = $ & $k_s \cdot (2^{k_s}\cdot (k_s-1)) \cdot 2$\\
Total NOT Gates & & $n \cdot (1 + \delta_g) \cdot \ell_N$ \\
Total Toffoli Gates & & $n \cdot (1 + \delta_g) \cdot \ell_T$ \\
Ancillae & & $(2^l)\cdot k_s \cdot (k_s + (k_s - 2)) \cdot 2$\\
Circuit Depth & & $l\cdot(\ell_T d_T + \ell_N d)$\\
\hline
\end{tabular}
\end{center}
\end{table}



\section{Parallelized Adding}
\label{sec:add}

\begin{tabular}{|c|c|c||c|c|}
\hline
$a_i$ & $b_i$ & $c_i$ & $s_i$ & $c_o$\\
\hline
0 & 0 & 0 & 0 & 0\\
0 & 0 & 1 & 1 & 0\\
0 & 1 & 0 & 1 & 0\\
0 & 1 & 1 & 0 & 1\\
1 & 0 & 0 & 1 & 0\\
1 & 0 & 1 & 0 & 1\\
1 & 1 & 0 & 0 & 1\\
1 & 1 & 1 & 1 & 1\\
\hline
\end{tabular}

\subsection{Resource Requirements for Adding}



\section{Parallelized Phase Sharpening}
\label{sec:sharp}

\begin{tabular}{|c|c|c||c|c||c|}
\hline
$\beta_j$ & $\alpha_{j+1}$ & $\alpha_{j+2}$ & $\overline{\beta_j}$ & $\overline{\alpha_j\alpha_{j+1}\alpha_{j+2}}$ & $\alpha_j$\\
\hline
000 & 0 & 0 & $\frac{0}{8}$ & $\frac{0}{8}$ & 0\\
    & 0 & 1 &               & $\frac{1}{8}$ & 0\\
    & 1 & 0 &               &               & X\\
    & 1 & 1 &               & $\frac{3}{8}$ & 0\\
\hline
001 & 0 & 0 & $\frac{1}{8}$ & $\frac{0}{8}$ & 0\\
    & 0 & 1 &               & $\frac{1}{8}$ & 0\\
    & 1 & 0 &               & $\frac{2}{8}$ & 0\\
    & 1 & 1 &               &               & X\\
\hline
010 & 0 & 0 & $\frac{2}{8}$ &               & X\\
    & 0 & 1 &               & $\frac{1}{8}$ & 0\\
    & 1 & 0 &               & $\frac{2}{8}$ & 0\\
    & 1 & 1 &               & $\frac{3}{8}$ & 0\\
\hline
011 & 0 & 0 & $\frac{3}{8}$ & $\frac{4}{8}$ & 1\\
    & 0 & 1 &               &               & X\\
    & 1 & 0 &               & $\frac{2}{8}$ & 0\\
    & 1 & 1 &               & $\frac{3}{8}$ & 0\\
\hline
100 & 0 & 0 & $\frac{4}{8}$ & $\frac{4}{8}$ & 1\\
    & 0 & 1 &               & $\frac{5}{8}$ & 1\\
    & 1 & 0 &               &               & X\\
    & 1 & 1 &               & $\frac{3}{8}$ & 0\\
\hline
101 & 0 & 0 & $\frac{5}{8}$ & $\frac{4}{8}$ & 1\\
    & 0 & 1 &               & $\frac{5}{8}$ & 1\\
    & 1 & 0 &               & $\frac{6}{8}$ & 1\\
    & 1 & 1 &               &               & X\\
\hline
110 & 0 & 0 & $\frac{6}{8}$ &               & X\\
    & 0 & 1 &               & $\frac{5}{8}$ & 1\\
    & 1 & 0 &               & $\frac{6}{8}$ & 1\\
    & 1 & 1 &               & $\frac{7}{8}$ & 1\\
\hline
111 & 0 & 0 & $\frac{7}{8}$ & $\frac{0}{8}$ & 0\\
    & 0 & 1 &               &               & X\\
    & 1 & 0 &               & $\frac{6}{8}$ & 1\\
    & 1 & 1 &               & $\frac{7}{8}$ & 1\\
\hline
\end{tabular}

\subsection{Resource Requirements for Phase Sharpening}


\section{Conclusion}
\label{sec:conclude}


\bibliography{report}
\bibliographystyle{tocplain}

\end{document}
