\section{Quantum Adding}

Since adding is such a common
operation in Super-Kitaev and many other quantum algorithms, it's important
for us to make sure we can do these efficiently.
The original development of Super-Kitaev in \cite{ksv02} gives us two
ways to add efficiently.
The first way reduces the sum of three numbers ($a,b,c$) to the sum of
two encoded numbers ($u,v$) in constant depth, simply by encoding the
bitwise sum $a_i+b_i+c_i$ as a 2-bit element from $\{0,1,2,3\}$. We arbitrarily
choose $u_i$ to be the low-order bit and $v_i$ to be high-order.
This can be used to parallelize the sum of $m$ numbers in
a $\log_{3/2}m$-depth tree, with $m-2$ encoded addition operations total.

\begin{figure}
\caption{Kitaev encoded adding: u}
\end{figure}

\begin{figure}
\caption{Kitaev encoded adding: v}
\end{figure}

\begin{figure}
\caption{$\log_{3/2} m$-depth tree for encoded adding of $m$ numbers}
\end{figure}

The encoded adding procedure is out-of-place and requires two ancillae
(one to hold the output bit $u_i$ or $v_i$). In order to apply
Bennett's uncomputing trick to perform this operation in-place, we round
the number of ancillae up to three to match the number of inputs, and we
use 3 SWAP gates (consisting of 3 CNOTs each). Because $v_i$ is high-order,
it always gets shifted up one bit: $v_0 = 0$ and $v_n$ the high-order carry
bit is discarded for sums modulo $2^n$.

The resource counts of Super-Kitaev encoded adding of $m$ $n$-bit numbers
are shown below, both for in-place and out-of-place.

\begin{tabular}{|r|c|c|}
\hline
Resource      & ENC-ADD-IN      & ENC-ADD-OUT\\
Depth         & $9$             & $6$\\ 
Toffoli Gates & $12 (n-1) + 6$  & $6 (n-1) + 6$\\
CNOT Gates    & $21 (n-1) + 15$ & $6 (n-1) + 6$\\
NOT Gates     & $10 (n-1)$      & $5 (n-1)$\\
Ancillae      & $3n$            & $4n/3$\\ 
\hline
\end{tabular}

As usual, the resource counts of the equivalent out-of-place procedure
are smaller.

We can combine the final two encoded numbers
into a normal $(O(\log m) + n)$-bit number using the second way of adding.
This method uses parallelized
iteration of general finite automata (FA) to achieve $O(\log n)$-depth circuits
for $O(n)$ inputs. Many functions, most importantly binary
addition and sharpening estimated phase values to exponential precision,
can ingeniously be recast in this parallelized FA setting.
First, $n$ copies are created of the FA
transition function table. Each copy is then narrowed by its corresponding
input,
composing adjacent function tables in a binary tree hierarchy, and applying
these composed functions to ``skip ahead'' in the FA and compute
intermediate states in parallel.

However, while achieving good asymptotic bounds, this approach has some
disadvantages when applied to quantum adding: an overly-general narrowing
procedure, and the inability to add in place without resorting to
Bennett's uncomputing trick \cite{bennett}.
For these reasons, we use the quantum carry-lookahead
adder (QCLA) presented in \cite{dkrs}. It follows the spirit of parallel iteration
using classical carry-lookahead techniques, in essence ``hardcoding''
the narrowing and composing steps above for the carry bits generated
and propagated in binary addition.

The resource counts for QCLA on two $n$-bit numbers are shown below,
for both in-place and out-of-place addition. The output bits for
out-of-place addition are counted as ancillae for consistency.

\begin{tabular}{|r|c|c|}
\hline
Resource & QCLA-IN & QCLA-OUT\\
Depth    &

$\lfloor \log n \rfloor  + \lfloor \log (n-1) \rfloor +
\lfloor \log n/3 \rfloor + \lfloor \log (n-1)/3 \rfloor + 14$ &
$\lfloor \log n \rfloor + \lfloor \log n/3 \rfloor + 7$ \\

Toffoli Gates &
$10n - 3w(n) - 3w(n-1) - 3\lfloor \log n \rfloor -
3 \lfloor \log(n-1) \rfloor - 7$ &
$5n - 3w(n) - 3 \lfloor \log n \rfloor - 1$\\

CNOT Gates & $4n-5$ & $3n-1$\\

NOT Gates & $2n-2$ & $0$ \\

Ancillae & $2n - w(n) - \lfloor \log n \rfloor - 1$ & $n$\\
\hline
\end{tabular}

By combining Super-Kitaev's encoded adding with the QCLA, we can now
add $m$ numbers of $n$-bits each with a circuit of size $O(mn)$ and
depth $O(\log m + \log n)$.

We denote the final resource counts of adding as the following:

\begin{eqnarray*}
\text{ADD-IN}(m, n) & = & \text{QCLA-IN}(n) + \text{ENC-ADD-IN}(m,n)\\
\text{ADD-OUT}(m, n) & = & \text{QCLA-OUT}(n) + \text{ENC-ADD-OUT}(m,n)\\
\end{eqnarray*}