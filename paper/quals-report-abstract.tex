\section{Abstract}

Quantum compilers will be needed to implement algorithms on an
80-qubit quantum computer being constructed in the next five years.
Like digital computers, quantum
computers need compilers to approximate high-level descriptions of
an algorithm using a low-level, universal, machine-dependent
instruction set. This work contributes
a numerical comparison of the
resources needed to run two separate quantum compiling algorithms, along
with the underlying open source code.
The first is the well-known Solovay-Kitaev theorem which showed that
efficient quantum compiling was possible in theory, but with some large
performance overheads in practice. The other is a
lesser-known result known as Super-Kitaev, which optimizes the compiled
circuit depth using parallelization at the expense of more ancilla qubits
and a larger overall circuit size. Finally, we discuss some implications for
future quantum architectures and also analog computers
(for when they come back into
fashion).
