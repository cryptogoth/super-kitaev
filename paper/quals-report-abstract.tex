\section{Abstract}

Quantum compilers will be needed to implement algorithms on an
80-qubit quantum computer being constructed in the next five years.
Important applications include Shor's factoring algorithm for breaking
the RSA cryptosystem, unstructured search, solving random walks, and
the simulation of physical systems. Like digital computers, quantum
computers need compilers to approximate high-level algorithm
descriptions using a low-level, universal, machine-dependent
instruction set. The Solovay-Kitaev theorem shows that this compiling
can be done efficiently, but the lesser-known Super-Kitaev theorem
optimizes this result using a quantum runtime component and
parallelization, at the cost of a previously unknown overhead. For
both algorithms, we present a pedagogical review of correctness
proofs, open source implementations, and a performance comparison in
approximating a two-qubit controlled-rotation gate, used in the
Quantum Fourier Transform. Along the way, we calculate the resource
usage of some useful building blocks in their own right (parallel
iteration of finite automata, quantum adders, and quantum phase
estimation). Finally, we discuss some implications for future quantum
architectures and also analog computers (for when they come back into
fashion).